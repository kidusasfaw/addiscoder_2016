%%
%% Style file "borrowed" from Michel Goemans
%%
\documentclass[11pt]{article}

\usepackage{url,amsmath,amsthm,amsfonts,amssymb}

\def\ceil#1{\lceil #1 \rceil}
\def\floor#1{\lfloor #1 \rfloor}
\def\ang#1{\langle #1 \rangle}

\newtheorem{definition}{Definition}
\newtheorem{remark}{Remark}
\newtheorem{theorem}{Theorem}
\newtheorem{lemma}[theorem]{Lemma}
\newtheorem{corollary}[theorem]{Corollary}
\newtheorem{proposition}[theorem]{Proposition}
\newtheorem{claim}[theorem]{Claim}
\newtheorem{observation}{Observation}

\newcommand{\R}{{\mathbb R}}
\newcommand{\Var}{\hbox{Var}}
\newcommand{\Z}{{\mathbb Z}}
\newcommand{\Q}{{\mathbb Q}}
\newcommand{\C}{{\mathbb C}}
\newcommand{\N}{{\mathbb N}}

\newlength{\toppush}
\setlength{\toppush}{2\headheight}
\addtolength{\toppush}{\headsep}

\newcommand{\htitle}[2]{\noindent\vspace*{-\toppush}\newline\parbox{6.5in}
 {\large Addis Ababa University, Amist Kilo \hfill #1\newline
\hspace*{\fill}{\bf Algorithms and Programming for High Schoolers} \hspace*{\fill} \newline
\mbox{}\hrulefill\mbox{}}\vspace*{1ex}\mbox{}\newline
\begin{center}{\Large\bf #2}\end{center}}

\newcommand{\handout}[3]{\thispagestyle{empty}
 \markboth{ #2}{ #2}
 \pagestyle{myheadings}\htitle{\protect\ref{#1}}{#2}{#3}}

\setlength{\oddsidemargin}{0pt}
\setlength{\evensidemargin}{0pt}
\setlength{\textwidth}{6.5in}
\setlength{\topmargin}{0in}
\setlength{\textheight}{8.5in}

\newcommand{\eps}{\varepsilon}
\newcommand{\sq}[1]{\langle#1\rangle}

\begin{document}

\htitle{July 22, 2016}{Lecture 5}

\paragraph{Recursion: worked examples}

\paragraph{Ternary Search.}
The last example we're covering today is ternary search.  Binary
search is useful for the following problem: we are given a sorted list
L[0] $\le$ L[1] $\le \ldots \le$ L[n-1], and we would like to find out
which $i$ has L[i] == x for some input x.  One way is with a
\texttt{for} loop:

\begin{verbatim}
# L is sorted
def findX(L, x):
    for i in xrange(len(L)):
        if L[i] == x:
            return i
    return -1
\end{verbatim}

However, in the worst case the above code takes $\Theta(n)$ time.  It
is much better to find $x$ by binary search as in Lecture 3, where we
check the middle element to see whether it is too small or too big,
then recursively check the half where $x$ might possibly lie.

What if L is not increasing, but rather is decreasing until some
unknown point, then increasing?  To find an $x$ in such a list we could
first find the position j where L switches from being decreasing to
increasing, then binary search in L[:j] and L[j:] separately to try to
find $x$.  So, how do we find this position $j$ where L switches
behavior?  To do this, we could use a an algorithm known as {\em
  ternary search}.

The idea behind ternary search is as follows.  First, we have to
assume that L never has the same value twice at two different
positions for this algorithm to work.  Now, for a \texttt{list} L of
length $n$ set \texttt{posA} = $n/3$ and \texttt{posB} = $2n/3$.  We
look at L[posA] and L[posB].  There are two cases: either L[posA] <
L[posB], or the other way around.  In the first case, it can be that
both posA and posB are to the right of the switching point j, or posA
is to the left of it and posB is to the right of it.  However, if
L[posA] < L[posB], it {\em cannot} be the case that both are to the
left of the switching point.  Thus, we can eliminate L[posB:] from
consideration.  Similarly in the case L[posA] > L[posB], it cannot be
the case that both posA and posB are to the right of the switching
point, so we can eliminate L[:posA+1].  In either case we eliminate
$1/3$rd of the possible entries and are thus left with only $2n/3$
possibilities.  The running time is thus the smallest $k$ such that
$(2/3)^k \cdot n \le 1$, so it is $\Theta(\log_{3/2} n)  =
\Theta(\log_2 n)$ (recall that $\log_a n = (1/\log_b a)\cdot \log_b
n$).

\begin{verbatim}
# find the switching point from decreasing to increasing
# in the list L[from:to+1]
def recurse(L, from, to):
    if from == to:
        return from
    # n items remaining
    n = to - from + 1
    posA = from + n/3
    posB = from + 2*n/3
    if L[posA] < L[posB]:
        return recurse(L, from, posB - 1)
    else:
        return recurse(L, posA + 1, to)

# find the switching point from decreasing to increasing
# in the list L
def ternarySearch(L, x):
    return recurse(L, 0, len(L) - 1)
\end{verbatim}

\paragraph{Nested brackets.}
Consider the following lists: [], [[]], [[[]]], [[[[]]]], etc.  The
first is the empty list, the second is a list containing the empty
list, the third is a list containing a list that contains the empty
list, etc.  We say that the first list in this sequence has nesting
level zero, and the second has nesting level 1, etc.  Write a function
which takes in the nesting level n and outputs the appropriate list.

\paragraph{Example solutions:}
With recursion:

\begin{verbatim}
def nest(n):
    if n == 0:
        return []
    return [nest(n-1)]
\end{verbatim}

Iteratively:

\begin{verbatim}
def nest(n):
    ans = []
    for i in xrange(n):
        ans = [ans]
    return ans
\end{verbatim}

\end{document}