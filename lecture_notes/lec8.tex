%%
%% Style file "borrowed" from Michel Goemans
%%
\documentclass[11pt]{article}

\usepackage{url,amsmath,amsthm,amsfonts,amssymb}
\usepackage{pgf}

\def\ceil#1{\lceil #1 \rceil}
\def\floor#1{\lfloor #1 \rfloor}
\def\ang#1{\langle #1 \rangle}

\newtheorem{definition}{Definition}
\newtheorem{remark}{Remark}
\newtheorem{theorem}{Theorem}
\newtheorem{lemma}[theorem]{Lemma}
\newtheorem{corollary}[theorem]{Corollary}
\newtheorem{proposition}[theorem]{Proposition}
\newtheorem{claim}[theorem]{Claim}
\newtheorem{observation}{Observation}

\newcommand{\R}{{\mathbb R}}
\newcommand{\Var}{\hbox{Var}}
\newcommand{\Z}{{\mathbb Z}}
\newcommand{\Q}{{\mathbb Q}}
\newcommand{\C}{{\mathbb C}}
\newcommand{\N}{{\mathbb N}}

\newlength{\toppush}
\setlength{\toppush}{2\headheight}
\addtolength{\toppush}{\headsep}

\newcommand{\htitle}[2]{\noindent\vspace*{-\toppush}\newline\parbox{6.5in}
 {\large Addis Ababa University, Amist Kilo \hfill #1\newline
\hspace*{\fill}{\bf Algorithms and Programming for High Schoolers} \hspace*{\fill} \newline
\mbox{}\hrulefill\mbox{}}\vspace*{1ex}\mbox{}\newline
\begin{center}{\Large\bf #2}\end{center}}

\newcommand{\handout}[3]{\thispagestyle{empty}
 \markboth{ #2}{ #2}
 \pagestyle{myheadings}\htitle{\protect\ref{#1}}{#2}{#3}}

\setlength{\oddsidemargin}{0pt}
\setlength{\evensidemargin}{0pt}
\setlength{\textwidth}{6.5in}
\setlength{\topmargin}{0in}
\setlength{\textheight}{8.5in}

\newcommand{\eps}{\varepsilon}
\newcommand{\sq}[1]{\langle#1\rangle}

\begin{document}

\htitle{July 27, 2016}{Lecture 8}

\paragraph{\Large More recursion/memoization examples:}

\paragraph{Example 1:}
Let \texttt{numWays}(n) be the number of ways to write a nonnegative
integer $n$ as the sum of positive integers.  For example, there are 8
ways of writing $4$: $1+1+1+1$, $2+1+1$, $1+2+1$, $1+1+2$, $2+2$,
$1+3$, $3+1$, and $4$.  One can show by induction that
\texttt{numWays}(n) = $2^{n-1}$, but let's see how to calculate it
using recursion and memoization.

\paragraph{Recursive implementation without memoization:}
\begin{verbatim}
def numWays(n):
    if n==0:
        return 1
    ans = 0
    for i in xrange(1, n+1):
        ans += numWays(n-i)
    return ans
\end{verbatim}

\paragraph{Recursive implementation with memoization:}
\begin{verbatim}
def memNumWays(n, mem):
    if n==0:
        return 1
    elif mem[n] != -1:
        return mem[n]
    mem[n] = 0
    for i in xrange(1, n+1):
        mem[n] += memNumWays(n-i, mem)
    return mem[n]

def numWays(n):
    mem = [-1]*(n+1)
    return memNumWays(n, mem)
\end{verbatim}

\paragraph{Example 2:}
What if we want to compute a function \texttt{distinctNumWays}(n)
which doesn't differentiate between different orderings of the same
sum?  For example, it treats $1+1+2$ and $2+1+1$ as the same sum.  So,
there would only be $5$ ways to sum up to the number $4$: $1+1+1+1$,
$1+2+2$, $2+2$, $1+3$, $4$.

We can calculate \texttt{distinctNumWays}(n) recursively as well, by
generating all ways of forming $n$ where the integers in the sum are
generating in nondecreasing order.  That is, we would not generate
$2+1+1$ or $1+2+1$ since the integers do not appear in nondecreasing
order; we would only generate $1+1+2$.  That way, we never count each
sum exactly once.

\paragraph{Recursive implementation without memoization:}
\begin{verbatim}
# how many ways are there to sum up to n, not counting different
# orderings of the sum, when the smallest number must be at least
# atLeast
def recurse(n, atLeast):
    if n==0:
        return 1
    ans = 0
    for i in xrange(atLeast, n+1):
        ans += recurse(n-i, i)
    return ans

def distinctNumWays(n):
    return recurse(n, 1)
\end{verbatim}

\paragraph{Recursive implementation with memoization:}
\begin{verbatim}
def recurse(n, atLeast, mem):
    if n==0:
        return 1
    elif mem[n][atLeast] != -1:
            return mem[n][atLeast]
    mem[n][atLeast] = 0
    for i in xrange(atLeast, n+1):
        mem[n][atLeast] += recurse(n-i, i, mem)
    return mem[n][atLeast]

def distinctNumWays(n):
    mem = [[-1]*(n+1)]*(n+1)
    for i in xrange(n+1):
        x = []
        for j in xrange(n+1):
            x += [-1]
        mem += [x]
    return recurse(n, 1, mem)
\end{verbatim}
\end{document}
