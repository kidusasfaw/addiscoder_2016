\documentclass[11pt]{article}

\usepackage{url,amsmath,amsthm,amsfonts,amssymb}
\usepackage[usenames]{color}
\usepackage{pgf}

\def\ceil#1{\lceil #1 \rceil}
\def\floor#1{\lfloor #1 \rfloor}
\def\ang#1{\langle #1 \rangle}

\newcommand{\TODO}[1]{\textcolor{red}{\textbf{todo:} \textit{#1}}}

\newtheorem{definition}{Definition}
\newtheorem{remark}{Remark}
\newtheorem{theorem}{Theorem}
\newtheorem{lemma}[theorem]{Lemma}
\newtheorem{corollary}[theorem]{Corollary}
\newtheorem{proposition}[theorem]{Proposition}
\newtheorem{claim}[theorem]{Claim}
\newtheorem{observation}{Observation}

\newcommand{\R}{{\mathbb R}}
\newcommand{\Var}{\hbox{Var}}
\newcommand{\Z}{{\mathbb Z}}
\newcommand{\Q}{{\mathbb Q}}
\newcommand{\C}{{\mathbb C}}
\newcommand{\N}{{\mathbb N}}

\newlength{\toppush}
\setlength{\toppush}{2\headheight}
\addtolength{\toppush}{\headsep}

\newcommand{\htitle}[2]{\noindent\vspace*{-\toppush}\newline\parbox{6.5in}
 {\large Addis Ababa University, Amist Kilo \hfill #1\newline
\hspace*{\fill}{\bf Algorithms and Programming for High Schoolers} \hspace*{\fill} \newline
\mbox{}\hrulefill\mbox{}}\vspace*{1ex}\mbox{}\newline
\begin{center}{\Large\bf #2}\end{center}}

\newcommand{\handout}[3]{\thispagestyle{empty}
 \markboth{ #2}{ #2}
 \pagestyle{myheadings}\htitle{\protect\ref{#1}}{#2}{#3}}

\setlength{\oddsidemargin}{0pt}
\setlength{\evensidemargin}{0pt}
\setlength{\textwidth}{6.5in}
\setlength{\topmargin}{0in}
\setlength{\textheight}{8.5in}

\newcommand{\eps}{\varepsilon}
\newcommand{\sq}[1]{\langle#1\rangle}

\begin{document}

\htitle{August 4, 2016}{Lecture 14}

\paragraph{\Large Graphs: worked examples.}

\paragraph{\Large Rainbow ride:}\ 

\smallskip

\noindent (Problem source: Kurukshetra 09 OPC, see
\texttt{http://www.spoj.pl/problems/RAINBOW/}).  A large family of $n$
people is going on vacation, and they are deciding who should go on a
certain roller coaster ride. Each person in the family likes some
other people in the family, and a person will only go on the ride if
everyone he likes and everyone who likes him also goes on the ride.
Each person has a weight, and the roller coaster can only support a
total weight of $W$.  Note: if two people don't like each other,
they're still allowed to go on the roller coaster together.  How do we
choose who to go on the ride so that the maximum number of people
participate?

How do we solve this problem?  First we build a graph where family
members are vertices, and there is an edge $(u,v)$ between two family
members if either $u$ likes $v$ or $v$ likes $u$.  Then if one family
member goes on the ride, everyone from his connected component must go
on the ride with him.  So, essentially we must choose which connected
components go on the ride so that the total weight of all connected
components we choose is at most $W$.  
We can first find the connected components and their total sizes and
weights by either DFS or BFS.  Then this
becomes exactly the knapsack
problem from Exercise 5, Lab 6.  We have some number $m$ of items
(each item is a connected component of the family graph), with
item $i$ having value $v_i$ (the number of people in that component)
and weight $w_i$ (the total weight of the component).  We must choose
which items to put in our knapsack of weight $W$ (i.e.\ the roller
coaster ride) so as to maximize the total value.  This can be solved
in time $\Theta(mW)$ using memoization; see the Lab 6 solutions for
details. \TODO{fix this reference?}

\paragraph{\Large Perimeter:}
You are given a 2D-image drawn using 10 colors. Each color is
represented by an
integer from 0 to 9. The image is described by giving the color of
each pixel in the image. Two
pixels are adjacent if one is immediately to the left of the other or
immediately above the other.
The image consists of many objects, and two pixels are part of the
same object if they are adjacent
and have the same color. The {\em perimeter} of an object is the
number of pixels on the boundary of the object; that is, the number of
pixels either bordering another object, or bordering one of the four
edges of the image.  Given a pixel in an image, calculate the
perimeter of the object it lies in.

\paragraph{Example solution:}
This problem can be solved using any one of depth-first search or
breadth-first search.  Create a graph where pixels are vertices, and
two vertices have an edge between them if the pixels are adjacent.
Then, we must explore the connected component of the starting pixel
and add one for each vertex we encounter which is either on the edge
or is adjacent to a vertex which is a part of a different object.

\begin{verbatim}
def isBorder(image, x, y):
    # check if (x,y) is on one of the four borders of the image
    if x==0 or x+1==len(image) or y==0 or y+1==len(image[0]):
        return 1

    # check if (x,y) borders a pixel with a different color
    dx = [0, 0, 1, -1]
    dy = [1, -1, 0, 0]
    for i in xrange(4):
        nx = x + dx[i]
        ny = y + dy[i]
        if image[x][y]!=image[nx][ny]:
            return 1
    
    # (x,y) is not on the border
    return 0

def recurse(image, x, y, seen):
    ans = isBorder(image, x, y)
    seen[x][y] = True
    dx = [0, 0, 1, -1]
    dy = [1, -1, 0, 0]
    for i in xrange(4):
        nx = x + dx[i]
        ny = y + dy[i]
        if nx>=0 and nx<len(image) and ny>=0 and ny<len(image[0]):
            if image[nx][ny]==image[x][y] and not seen[nx][ny]:
                ans += recurse(image, nx, ny, seen)
    return ans

# recursive DFS implementation
# image is a list of lists, giving pixel colors
# we should compute the answer for the object containing pixel (x,y)
def perimeter(image, x, y):
    seen = []
    for i in xrange(len(image)):
        seen += [[False]*len(image[0])]
    return recurse(image, x, y, seen)
\end{verbatim}

Here is another implementation which uses an explicit stack.  The
\texttt{isBorder} function is the same as last time, so we don't
repeat it here.

\begin{verbatim}
def visit(x, y, stack, seen):
    seen[x][y] = True
    stack += [[x, y]]

# DFS implementation with explicit stack
# image is a list of lists, giving pixel colors
# we should compute the answer for the object containing pixel (x,y)
def perimeter(image, x, y):
    seen = []
    for i in xrange(len(image)):
        seen += [[False]*len(image[0])]
    ans = 0
    stack = []
    visit(x, y, stack, seen)
    while len(stack) > 0:    
        p = stack.pop()
        a = p[0]
        b = p[1]       
        ans += isBorder(image, a, b)
        dx = [0, 0, 1, -1]
        dy = [1, -1, 0, 0]
        for i in xrange(4):
            nx = a + dx[i]
            ny = b + dy[i]
            if nx>=0 and nx<len(image) and ny>=0 and ny<len(image[0]):
                if image[nx][ny]==image[a][b] and not seen[nx][ny]:
                    visit(nx, ny, stack, seen)
    return ans
\end{verbatim}

\paragraph{\Large Stepping on Nails:}
You're in a 2D-room which is n by m meters.  
The room is divided into
squares that are each 1 square meter each.  
The bottom-left square is (0,0), and the top-right is (n-1,m-1).
You start in square (x,y) and are trying to get to square (u,v).  
From each square you can go to 8 other squares: up, down, left, right,
and also the four diagonals.
Some
squares have nails on the floor, and if you step on it, it hurts!  You
don't mind stepping on a few nails, but if you step on more than $N$
nails then you'll need to go to the hospital.  Calculate the shortest
way of getting to (u,v) without going to the hospital.  
If it is impossible, return -1.
The starting and ending locations will not have nails.

\paragraph{Example solution:}
We create a graph where each vertex represents (x,y,t): the square
$(x,y)$ we are at, and the number of nails $t$ we have stepped on so
far (we only have to consider $t \le N$).  Each vertex has eight edges
leading to the adjacent vertices,
where t either stays the same or increases by 1, depending on whether
we stepped onto a nail.  We should return the minimum distance to
$(u,v,t)$ over all $0\le t \le N$, which we can find using BFS.  If we
never managed to reach such a $(u,v,t)$, the answer is $-1$.

\begin{verbatim}
from collections import deque
def visit(x, y, t, queue, distance, D):
    distance[x][y][t] = D
    queue += [[x,y,t]]

# room[i][j] is True if there's a nail and False otherwise
def fastestRoute(room, x, y, u, v, N):
    # the makeArray function from Lecture 16
    distance = makeArray([room, room[0], N+1], -1)
    queue = deque()
    visit(x, y, 0, queue, distance, 0)
    while len(queue) > 0:
        p = queue.popleft()
        a = p[0]
        b = p[1]
        t = p[2]
        if a==u and b==v:
            return distance[a][b][t]
        for dx in xrange(-1, 2):
            for dy in xrange(-1, 2):
                if dx==0 and dy==0: continue
                nx = a + dx
                ny = b + dy
                if nx>=0 and nx<len(room) and ny>=0 and ny<len(room[0]):
                    nt = t
                    if room[nx][ny]: nt += 1
                    if nt <= N:
                        visit(nx, ny, nt, queue, distance, distance[a][b][t]+1)
    return -1
\end{verbatim}

\paragraph{\Large Catching a cockroach:}
You and a cockroach are in a room. You hate cockroaches, so you want
to catch it and kill it.  You start at location (x,y) and the
cockroach starts at (u,v).  
The room is n x m meters, and the bottom-left square is (0,0), and the
top-right is (n-1,m-1).
Cockroaches are simple creatures, and you
know how the cockroach behaves: you are given two lists dx and dy of
moves so
that you know that in one step, the cockroach moves to (u + dx[0], v +
dy[0]), then from there moves to (u + dx[0] + dx[1], v + dy[0], v +
dy[1]), etc.  In other words, the cockroach moves by (dx[i], dy[i]) in
the ith step.  If any particular step would make the cockroach run
into a wall, the cockroach just stays where he is in that step.  If he
reaches the end of his list of moves, he cycles back and does the 0th
move again.  What is the minimum amount of time required to catch the
cockroach? If any step, we can move either up, down, left, or right.
We can also just choose to stay where we are.

\paragraph{Example solution:}
We create a graph where the vertices represent (a,b,c,d,i), where our
location is (a,b), the cockroach's location is (c,d), and the
cockroach's next move is (dx[i], dy[i]).  We need to return the
minimum distance to any vertex where (a,b) equals (c,d), and i can be
any integer $0\le i < \mathrm{len}(dx)$.  This can be done using
breadth first search.

\begin{verbatim}
from collections import deque
def visit(x, y, u, v, i, queue, distance, D):
    distance[x][y][u][v] = D
    queue += [[x,y,u,v,i]]

# The room is only n x m meters
def catchRoach(x, y, u, v, n, m, dx, dy):
    # the makeArray function from Lecture 16
    distance = makeArray([n, m, n, m, len(dx)], -1)
    queue = deque()
    visit(x, y, u, v, 0, queue, distance, 0)
    while len(queue) > 0:
        p = queue.popleft()
        a = p[0]
        b = p[1]
        c = p[2]
        d = p[3]
        i = p[4]
        if a==c and b==d:
            return distance[a][b][c][d][i]
        cx = [0, 0, -1, 1, 0]
        cy = [-1, 1, 0, 0, 0]
        # calculate the roach's new location
        nrx = c + dx[i]
        nry = d + dy[i]
        # the roach can't leave the room
        if nrx<0 or nrx==n or nry<0 or nry==m:
            nrx = c
            nry = d
        for j in xrange(5):
            nx = a + cx[j]
            ny = b + cy[j]
            if nx>=0 and nx<n and ny>=0 and ny<m:
                visit(nx, ny, nrx, nry, (i+1)%len(dx),
                      queue, distance, distance[a][b][c][d][i]+1)
\end{verbatim}

\end{document}