%%
%% Style file "borrowed" from Michel Goemans
%%
\documentclass[11pt]{article}

\usepackage{url,amsmath,amsthm,amsfonts,amssymb}

\def\ceil#1{\lceil #1 \rceil}
\def\floor#1{\lfloor #1 \rfloor}
\def\ang#1{\langle #1 \rangle}

\newtheorem{definition}{Definition}
\newtheorem{remark}{Remark}
\newtheorem{theorem}{Theorem}
\newtheorem{lemma}[theorem]{Lemma}
\newtheorem{corollary}[theorem]{Corollary}
\newtheorem{proposition}[theorem]{Proposition}
\newtheorem{claim}[theorem]{Claim}
\newtheorem{observation}{Observation}

\newcommand{\R}{{\mathbb R}}
\newcommand{\Var}{\hbox{Var}}
\newcommand{\Z}{{\mathbb Z}}
\newcommand{\Q}{{\mathbb Q}}
\newcommand{\C}{{\mathbb C}}
\newcommand{\N}{{\mathbb N}}

\newlength{\toppush}
\setlength{\toppush}{2\headheight}
\addtolength{\toppush}{\headsep}

\newcommand{\htitle}[2]{\noindent\vspace*{-\toppush}\newline\parbox{6.5in}
 {\large Sports Academy \hfill #1\newline
\hspace*{\fill}{\bf Algorithms and Programming for High Schoolers} \hspace*{\fill} \newline
\mbox{}\hrulefill\mbox{}}\vspace*{1ex}\mbox{}\newline
\begin{center}{\Large\bf #2}\end{center}}

\newcommand{\handout}[3]{\thispagestyle{empty}
 \markboth{ #2}{ #2}
 \pagestyle{myheadings}\htitle{\protect\ref{#1}}{#2}{#3}}

\setlength{\oddsidemargin}{0pt}
\setlength{\evensidemargin}{0pt}
\setlength{\textwidth}{6.5in}
\setlength{\topmargin}{0in}
\setlength{\textheight}{8.5in}

\newcommand{\eps}{\varepsilon}
\newcommand{\sq}[1]{\langle#1\rangle}

\begin{document}

\htitle{July 19, 2016}{Lecture 2}

\paragraph{\Large if Statements:}

The \texttt{if} statement allows you to only conditionally execute
some code block, conditioned on some expression evaluating to
\texttt{True}.

\begin{verbatim}
if BOOL_EXPR:
    CODE_BLOCK
elif BOOL_EXPR:
    CODE_BLOCK
...
elif BOOL_EXPR:
    CODE_BLOCK
else:
    CODE_BLOCK
\end{verbatim}


In the above code, exactly one of the code blocks is executed,
corresponding to the first \texttt{BOOL\_EXPR} which evaluates to
\texttt{True} (or the final code block corresponding to the
\texttt{else} in the case that none of the \texttt{BOOL\_EXPR}
evaluates to \texttt{True}).
The \texttt{elif} and \texttt{else} statements are optional.

\paragraph{Example:}
\begin{verbatim}
def printSign(n):
    if n < 0:
        print `Negative'
    elif n > 0:
        print `Positive'
\end{verbatim}

Now, if we were to execute \texttt{printSign}(-1), `Negative' would
be printed, and similarly \texttt{printSign}(1) would print
`Positive'.  Calling \texttt{printSign}(0) would result in nothing
being printed.

\paragraph{\Large for Statements:}
The \texttt{for} statement allows you to iterate over data in Python
(for example, iterating over items in a \texttt{list}, or characters
in \texttt{str}).

\begin{verbatim}
for var in v:
   CODE_BLOCK
\end{verbatim}

The expression \texttt{v} above should evaluate to something
iterable.

\paragraph{Example:}
\begin{verbatim}
fruits = [`orange', `pineapple', `banana', `mango']
pluralFruits = []
for x in fruits:
    plural = x + `s'
    pluralFruits += [plural]
\end{verbatim}

\noindent The above code segment would cause \texttt{favoriteFruits}
to equal [`oranges', `pineapples', `bananas', `mangos'].

\paragraph{\Large while Statements:}
The \texttt{while} statement allows you to repeatedly execute a code
block as long as some \texttt{bool} expression evaluates to
\texttt{True}.

\begin{verbatim}
while BOOL_EXPR:
   CODE_BLOCK
\end{verbatim}

\paragraph{Example:}
\begin{verbatim}
x = []
y = 0
while y < 10:
    x += [y]
    y += 1
\end{verbatim}

\noindent The above code segment would cause \texttt{x} to equal
[0, 1, 2, 3, 4, 5, 6, 7, 8, 9].

\paragraph{\Large break and continue:}

Sometimes you might want to stop iterating in a \texttt{for} or
\texttt{while} early, or just skip some particular iteration.  The
\texttt{break} and \texttt{continue} statements are useful for this.
\texttt{break} exits the loop early, and \texttt{continue} moves back
to the beginning of the loop.

\paragraph{Example:}

Both of these code examples print only the odd numbers between 0 and
5.

\begin{verbatim}
# Example with for loop
myList = [0, 1, 2, 3, 4, 5, 6, 7, 8, 9, 10]
for x in myList:
    if x > 5:
        break
    elif x%2 == 0:
        continue
    else:
        print x
\end{verbatim}

\begin{verbatim}
# Example with while loop
x = 0
while True:
    if x > 5:
        break
    elif x%2 == 0:
        x += 1
        continue
    else:
        print x
    x += 1
\end{verbatim}

\paragraph{\Large Other useful functions:}
It will be helpful for today's lab to know the following functions.

\begin{itemize}
\item \texttt{len}(x) returns the length of an iterable data type
  (such as a \texttt{str} or \texttt{list}) as an \texttt{int}.  For
  example, \texttt{len}(`abc') and \texttt{len}([`a', `b', `c']) 
  both evaluate to 3.  \texttt{len}([`a', [`b', `c', `d']]) 
  evaluates to 2.
\item \texttt{range}(x) returns a list of \texttt{int}s from $0$ to
  $x-1$.  For example, \texttt{range}(5) returns [0, 1, 2, 3, 4].  You
  can also give \texttt{range} a starting value (\texttt{range}(2, 5)
  returns [2, 3, 4]) and a ``skip-by'' value (\texttt{range}(0, 10, 2)
  returns [0, 2, 4, 6, 8]).
\item \texttt{xrange}(x) is similar to \texttt{range}(x), except that
  it does not actually return a list, but rather returns an object
  which can be iterated over and has the same values as if
  \texttt{range} had been called.  We have not spoken about objects
  yet, so do not worry too much about what that means, but the main
  point is that \texttt{xrange} only generates the next value as you
  need it without ever explicitly storing the entire \texttt{list} in
  memory, whereas \texttt{range} would explicitly store the list.
  Thus, \texttt{xrange} can be helpful if you know your code will
  \texttt{break} early in a long sequence.  Consider the following
  examples:

\begin{verbatim}
# Example 1
for x in range(1000000000):
    if x == 6:
        break
    print x

# Example 2
for x in xrange(1000000000):
    if x == 6:
        break
    print x
\end{verbatim}

In both cases the numbers $0,1,2,3,4,5$ are printed, but the first
code example is almost a billion times slower because \texttt{range}
will actually generate a list of size one billion at the beginning of
the loop, whereas \texttt{xrange} just generates the next number as
it is needed.
\end{itemize}

\end{document}
