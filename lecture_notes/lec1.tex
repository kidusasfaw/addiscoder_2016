%%
%% Style file "borrowed" from Michel Goemans
%%
\documentclass[11pt]{article}

\usepackage{url,amsmath,amsthm,amsfonts,amssymb}

\def\ceil#1{\lceil #1 \rceil}
\def\floor#1{\lfloor #1 \rfloor}
\def\ang#1{\langle #1 \rangle}

\newtheorem{definition}{Definition}
\newtheorem{remark}{Remark}
\newtheorem{theorem}{Theorem}
\newtheorem{lemma}[theorem]{Lemma}
\newtheorem{corollary}[theorem]{Corollary}
\newtheorem{proposition}[theorem]{Proposition}
\newtheorem{claim}[theorem]{Claim}
\newtheorem{observation}{Observation}

\newcommand{\R}{{\mathbb R}}
\newcommand{\Var}{\hbox{Var}}
\newcommand{\Z}{{\mathbb Z}}
\newcommand{\Q}{{\mathbb Q}}
\newcommand{\C}{{\mathbb C}}
\newcommand{\N}{{\mathbb N}}

\newlength{\toppush}
\setlength{\toppush}{2\headheight}
\addtolength{\toppush}{\headsep}

\newcommand{\htitle}[2]{\noindent\vspace*{-\toppush}\newline\parbox{6.5in}
 {\large Sports Academy \hfill #1\newline
\hspace*{\fill}{\bf Algorithms and Programming for High Schoolers} \hspace*{\fill} \newline
\mbox{}\hrulefill\mbox{}}\vspace*{1ex}\mbox{}\newline
\begin{center}{\Large\bf #2}\end{center}}

\newcommand{\handout}[3]{\thispagestyle{empty}
 \markboth{ #2}{ #2}
 \pagestyle{myheadings}\htitle{\protect\ref{#1}}{#2}{#3}}

\setlength{\oddsidemargin}{0pt}
\setlength{\evensidemargin}{0pt}
\setlength{\textwidth}{6.5in}
\setlength{\topmargin}{0in}
\setlength{\textheight}{8.5in}

\newcommand{\eps}{\varepsilon}
\newcommand{\sq}[1]{\langle#1\rangle}

\begin{document}

\htitle{July 18, 2016}{Lecture 1}

Basic data types: \texttt{int}, \texttt{long}, \texttt{float},
\texttt{bool},
\texttt{str}, \texttt{list}. There are others 
we aren't covering yet.

\bigskip

\paragraph{Operators for numerical data types (\texttt{int,long,float}):}

\begin{center}

\begin{tabular}{|c|l|l|}
\hline
Operator & Description & Example\\
\hline
$+$ & add two values & $5+2$ gives $7$\\
\hline
$-$ & subtract two values & $5-2$ gives $3$\\
\hline
$*$ & multiply two values & $5*2$ gives $10$\\
\hline
$/$ & divide two values& $5.0/2.0$ gives $2.5$\\
& rounds down for ints & $5/2$ gives $2$\\
\hline
$\%$ & ``mod'', remainder after division& $5\% 2$ gives $1$\\
\hline
$**$ & exponentiation & $5**2$ gives $25$\\
\hline
$//$ & division with floor& $5.0//2.0$ gives $2.0$\\
& (rounds down to integer value) & $5//2$ gives $2$\\
\hline
\end{tabular}
\end{center}

\paragraph{Operators for bools:}

\begin{center}

\begin{tabular}{|c|l|l|}
\hline
Operator & Description & Example\\
\hline
\texttt{and} & logical and & \texttt{True and True} gives \texttt{True}\\
&  & \texttt{True and False} gives \texttt{False}\\
&  & \texttt{False and True} gives \texttt{False}\\
&  & \texttt{False and False} gives \texttt{False}\\
\hline
\texttt{or} & logical or & \texttt{True or True} gives \texttt{True}\\
&  & \texttt{True or False} gives \texttt{True}\\
&  & \texttt{False or True} gives \texttt{True}\\
&  & \texttt{False or False} gives \texttt{False}\\
\hline
\texttt{not} & logical not & \texttt{not True} gives \texttt{False}\\
&  & \texttt{not False} gives \texttt{True}\\
\hline
\end{tabular}
\end{center}

\paragraph{Operators and indexing for lists:}

\begin{center}

\begin{tabular}{|c|l|l|}
\hline
Operator & Description & Example\\
\hline
$+$ & concatenate two lists & $[5]+[2]$ gives $[5,2]$\\
\hline
$*$ & repeat a list several times (\texttt{list}*\texttt{int} or \texttt{int}*\texttt{list}) & $[1,2]*3$ gives $[1,2,1,2,1,2]$\\
\hline
$[i]$ & gives $i$th element in the list & ['a','b','c'][0] gives 'a'\\
\hline
$[i:j]$ & take sublist from element $i$ to $j-1$ &
['a','b','c','d','e'][1:3] gives ['b','c']\\
\hline
\end{tabular}
\end{center}

\paragraph{Operators and indexing for strings:}

\begin{center}

\begin{tabular}{|c|l|l|}
\hline
Operator & Description & Example\\
\hline
$+$ & concatenate two strings & `hi' + ` there' gives `hi there'\\
\hline
$*$ & repeat a string several times (\texttt{str}*\texttt{int} or \texttt{int}*\texttt{str}) & `hi'*3 gives `hihihi'\\
\hline
$[i]$ & gives $i$th character in the string & 'abcdefg'[3] gives 'd'\\
\hline
$[i:j]$ & take substring from character $i$ to $j-1$& 'abcdefg'[3:5]
gives 'de'\\
\hline
\end{tabular}
\end{center}

\paragraph{Other stuff:}

\begin{center}
Comparison Operators\\
(in below table,  the description assumes $a$ OPERATOR $b$)

\bigskip

\begin{tabular}{|c|l|l|}
\hline
Operator & Description & Example\\
\hline
$==$ & True if $a$ equals $b$ & $5==2$ is False\\
\hline
$!=$ & True if $a$ doesn't equal $b$ & $5!=2$ is True\\
\hline
$<>$ & Same as $!=$ & $5<>2$ is True\\
\hline
$>$ & True if $a$ is greater than $b$ & $5>2$ is True\\
\hline
$<$ & True if $a$ is less than $b$ & $5<2$ is False\\
\hline
$>=$ & True if $a$ is greater than or equal to $b$ & $5>=2$ is True\\
\hline
$<=$ & True if $a$ is less than or equal to $b$ & $5<=2$ is False\\
\hline
\end{tabular}

\bigskip

Assignment Operators\\
Left hand side is a variable, call it ``x'', and right hand side is
an expression, call it ``b''.\\
(In examples, assume $x$ has the value $17$)


\bigskip

\begin{tabular}{|c|l|l|}
\hline
Operator & Description & Example\\
\hline
$=$ & Sets $x$ to the result of $b$ & $x = 5+2$ sets $x$ to $7$\\
\hline
$+=$ & Sets $x$ to $x+b$ & $x += 5+2$ sets $x$ to $14$\\
\hline
$-=$ & Sets $x$ to $x-b$ & $x -= 5+2$ sets $x$ to
$7$\\
\hline
$*=$ & Sets $x$ to $x*b$ & $x *= 5+2$ sets $x$ to
$49$\\
\hline
$/=$ & Sets $x$ to $x/b$ & $x /= 5+3$ sets $x$ to $6$\\
\hline
$\%=$ & Sets $x$ to $x\% b$ & $x \%= 5+2$ sets $x$ to $6$\\
\hline
$**=$ & Sets $x$ to $x**b$& $x**=2$ sets $x$ to $36$\\
\hline
$//=$ & Sets $x$ to $x//b$ & $x //= 5+2$
sets $x$ to $5\\
\hline
\end{tabular}

\bigskip

Operator Precedence, from highest precedence to lowest\\
(remember you can override operator precedence using parentheses)

\bigskip

\begin{tabular}{|c|}
\hline
Operation\\
\hline
$**$ \\
\hline
$*\ /\ \%\ //$ \\
\hline
$+ -$ \\
\hline
$<=\ <\ >\ >=$ \\
\hline
$<>\ ==\ \ !=$ \\
\hline
$=\ \%=\ /=\ //=\ -=\ +=\ *=\ **=$ \\
\hline
\texttt{not or and}\\
\hline
\end{tabular}
\end{center}

\paragraph{Example of defining and evaluating a function:}

\bigskip

\begin{verbatim}
>>> def addTwo(n):
...     return n+2
... 
>>> addTwo(5)
7
\end{verbatim}

\noindent Note: variables can point to functions.

\begin{verbatim}
>>> x = addTwo
>>> x(2)
4
\end{verbatim}


\end{document}
