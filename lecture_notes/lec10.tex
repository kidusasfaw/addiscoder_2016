\documentclass[11pt]{article}

\usepackage{url,amsmath,amsthm,amsfonts,amssymb}
\usepackage{pgf}

\def\ceil#1{\lceil #1 \rceil}
\def\floor#1{\lfloor #1 \rfloor}
\def\ang#1{\langle #1 \rangle}

\newtheorem{definition}{Definition}
\newtheorem{remark}{Remark}
\newtheorem{theorem}{Theorem}
\newtheorem{lemma}[theorem]{Lemma}
\newtheorem{corollary}[theorem]{Corollary}
\newtheorem{proposition}[theorem]{Proposition}
\newtheorem{claim}[theorem]{Claim}
\newtheorem{observation}{Observation}

\newcommand{\R}{{\mathbb R}}
\newcommand{\Var}{\hbox{Var}}
\newcommand{\Z}{{\mathbb Z}}
\newcommand{\Q}{{\mathbb Q}}
\newcommand{\C}{{\mathbb C}}
\newcommand{\N}{{\mathbb N}}

\newlength{\toppush}
\setlength{\toppush}{2\headheight}
\addtolength{\toppush}{\headsep}

\newcommand{\htitle}[2]{\noindent\vspace*{-\toppush}\newline\parbox{6.5in}
 {\large Addis Ababa University, Amist Kilo \hfill #1\newline
\hspace*{\fill}{\bf Algorithms and Programming for High Schoolers} \hspace*{\fill} \newline
\mbox{}\hrulefill\mbox{}}\vspace*{1ex}\mbox{}\newline
\begin{center}{\Large\bf #2}\end{center}}

\newcommand{\handout}[3]{\thispagestyle{empty}
 \markboth{ #2}{ #2}
 \pagestyle{myheadings}\htitle{\protect\ref{#1}}{#2}{#3}}

\setlength{\oddsidemargin}{0pt}
\setlength{\evensidemargin}{0pt}
\setlength{\textwidth}{6.5in}
\setlength{\topmargin}{0in}
\setlength{\textheight}{8.5in}

\newcommand{\eps}{\varepsilon}
\newcommand{\sq}[1]{\langle#1\rangle}

\begin{document}

\htitle{July 29, 2016}{Lecture 10}

\paragraph{\Large Memoization: worked examples.}

\paragraph{Longest Common Subsequence:}
Given a \texttt{str} $s = s[0]s[1]\ldots s[n-1]$, a {\em subsequence}
of $s$ is a \texttt{str} $s[i_1]s[i_2]\ldots s[i_r]$ with $0 \le i_1 <
i_2 \ldots < i_r \le n-1$. That is, you form a subsequence by going
from left to right along the \texttt{str} $s$ and taking some subset
of letters (including possibly no letters, in which case you'd get the
empty \texttt{str}).

Now, the algorithmic problem we will consider is the following:
computing the length of the longest common subsequence (LCS) of two
strings
$s$ and $t$.  The idea is to use recursion, which we can speed up
using memoization.  When trying to form a LCS
for $s$,$t$, let us consider the first letters of both $s$ and $t$.
We have three options.  Either both the first letter of $s$ and the
first letter of $t$ are in the LCS (in which
case these letters must be equal), or one of them {\em isn't
} in the LCS.  This is of course assuming $s$ and $t$ both have a
first letter; if one of them is the empty string, then their LCS is
the empty string, which has length $0$.  Thus we have the following:

$$\mathrm{LCS}(s,t) =\begin{cases} 0 \ &
  \mathrm{if}\ \mathrm{len}(s)=0\ \mathrm{or}\ \mathrm{len}(t) = 0
\\ \max\{1 + \mathrm{LCS}(s[1:],t[1:]), \mathrm{LCS}(s[1:],t),
\mathrm{LCS}(s,t[1:])\} \ & \mathrm{if}\ s[0] == t[0]\\
\max\{\mathrm{LCS}(s[1:],t),
\mathrm{LCS}(s,t[1:])\}\ &  \mathrm{otherwise} \end{cases} .$$

(In fact it's not too hard to show that if $s[0]==t[0]$ then the best
option is always to take both $s[0]$ and $t[0]$ in the LCS, and thus
the the middle case can just have one option instead of three.)

The above can then be implemented in code recursively:

\begin{verbatim}
def lcs(s, t):
    if len(s)==0 or len(t)==0:
        return 0
    ans = lcs(s[1:], t)
    ans = max(ans, lcs(s, t[1:])
    if s[0] == t[0]:
        ans = max(ans, 1 + lcs(s[1:], t[1:]))
    return ans
\end{verbatim}

The above code though is quite slow and can be sped up via
memoization.

\begin{verbatim}
# compute the LCS between s[atS:] and t[atT:]
def recurse(s, t, atS, atT, mem):
    if atS==len(s) or atT==len(t):
        return 0
    elif mem[atS][atT] != -1:
        return mem[atS][atT]
    ans = lcs(s, t, atS + 1, atT, mem)
    ans = max(ans, lcs(s, t, atS, atT + 1, mem)
    if s[atS] == t[atT]:
        ans = max(ans, 1 + lcs(s, t, atS + 1, atT + 1, mem))
    mem[atS][atT] = ans
    return ans

def lcs(s, t):
    mem = []
    for i in xrange(len(s)):
        mem += [[-1]*len(t)]
    return recurse(s, t, 0, 0, mem)
\end{verbatim}

The running time of the memoized version is $\Theta(\mathrm{len}(s)
\cdot \mathrm{len}(t))$.

\paragraph{Edit Distance:}
Have you ever misspelled a word when searching on Google, and Google
asks you ``Did you mean to search for $\ldots$?'', with the correct
spelling given?  Or have you ever used a spellchecker in Microsoft
Word or some other word processor which gives you suggestions for how
to correct your misspellings?  For these and other applications, it is
useful to have an algorithm for computing the {\em edit distance}
between two strings.  The edit distance between strings $s$ and $t$ is
the minimum number of ``changes'' that one can make to $s$ to turn it
into $t$.  The allowed changes are one of the following: (1) insert
any character at any position in $s$, (2) delete any character from
$s$, and (3) change any character in $s$ to a different character.
For example, consider the misspelling ``fantom'' of ``phantom''.  Three
changes can be made to fantom to turn it into phanton: delete the
``f'', then insert the ``p'' and ``h''.  However, a cheaper way is to
replace the ``f'' with an ``h'' then insert the ``p'' in the
beginning, and in fact this is the cheapest way, so the edit distance
of those two strings is $2$.

How does one compute the edit distance between two strings $s$ and
$t$?  Again this can be done using recursion and memoization.  If $s$
is the empty string, we must insert $\mathrm{len}(t)$ letters into $s$ to
change it into $t$.  If $t$ is the empty string, we must delete
$\mathrm{len}(s)$ letters to turn $s$ into the empty string.
Otherwise, we have a few options.  If $s[0] == t[0]$, we have the
option of just trying to turn $s[1:]$ into $t[1:]$.  Otherwise, we can
insert $t[0]$ right before $s$ then try to turn $s$ into $t[1:]$.  We
can also delete $s[0]$ then try to turn $s[1:]$ into $t$.  Finally, we
can also change $s[0]$ into $t[0]$ then try to change $s[1:]$ into
$t[1:]$.  Let ED represent edit distance.  Then,

$$\mathrm{ED}(s,t) =\begin{cases} \mathrm{len}(t) \ &
  \mathrm{if}\ \mathrm{len}(s)=0
\\ 
\mathrm{len}(s) \ &
  \mathrm{if}\ \mathrm{len}(t)=0
\\\min\{\mathrm{ED}(s[1:],t[1:]), 1 + \mathrm{ED}(s[1:],t),
1 + \mathrm{ED}(s,t[1:])\} \ & \mathrm{if}\ s[0] == t[0]\\
\max\{1 + \mathrm{ED}(s[1:],t[1:]), 1+\mathrm{ED}(s[1:],t),
1+\mathrm{ED}(s,t[1:])\}\ &  \mathrm{otherwise} \end{cases} .$$

In code:

\begin{verbatim}
def ED(s, t):
    if len(s)==0:
        return len(t)
    elif len(t)==0:
        return len(s)
    elif s[0] == t[0]:
        return min(ED(s[1:],t[1:]), 1 + min(ED(s[1:], t), ED(s, t[1:])))
    else:
        return 1 + min(ED(s[1:],t[1:]), min(ED(s[1:], t), ED(s, t[1:])))
\end{verbatim}

The above implementation can be sped up using memoization, as follows.

\begin{verbatim}
# return the edit distance between s[atS:] and t[atT:]
def recurse(s, t, atS, atT, mem):
    if atS == len(s):
        return len(t)
    elif atT == len(t):
        return len(s)
    ans = 1 + min(recurse(s, t, atS + 1, atT, mem), recurse(s, t, atS, atT + 1, mem))
    if s[0] == t[0]:
        ans = min(ans, recurse(s, t, atS + 1, atT + 1, mem))
    else:
        ans = min(ans, 1 + recurse(s, t, atS + 1, atT + 1, mem))
    mem[atS][atT] = ans
    return ans

def ED(s, t):
    mem = []
    for i in xrange(len(s)):
        mem += [[-1]*len(t)]
    return recurse(s, t, 0, 0, mem)
\end{verbatim}

\paragraph{Egg Dropping:}
Suppose we live in a building with $H$ floors, and we are trying to
figure out the highest floor from which we can drop an egg such that
the egg won't break.  We make a few assumptions:

\begin{itemize}
\item All eggs behave the same way.  That is, if one egg would or
  wouldn't drop at any given floor, then neither would any other.
\item If an egg would break being dropped from some floor, then it
  would also break being dropped from a higher floor.
\item If an egg wouldn't break being dropped from some floor, then it
  also wouldn't break being dropped from a lower floor.
\item If an egg breaks from a fall, we can't reuse it. If it doesn't
  break, we can reuse it.
\end{itemize}

Now, suppose we have $N$ eggs and would like to know: what is the
minimum number of egg drop tests we have to do to find the highest
floor from which it is safe to drop eggs?

Well, if we have $0$ eggs and $H>0$, the
task is impossible; we'll represent this by the answer $\infty$.  If
$H=0$, then the answer is $0$ tests.  If $N=1$, then we have no other
option than to start at the lowest floor and try floor by floor going
upwards, thus requiring $H$ tests.  Otherwise, we can choose a floor
$x$ to drop the egg from.  Let $f(N, H)$ be the worst-case minimum
number of
tests required when we have $N$ eggs and $H$ consecutive floors left
to test.  Then, if our first egg drop is at floor $x$, either
it will break and we have to do $f(N-1, x-1)$ more tests, or it won't
and
we'll have to do $f(N, H - x)$ more tests.  Thus, in the worst case,
we
will have to do $\max\{f(N-1, H-1), f(N, H-x)\}$ extra tests.  We can
choose $x$ to minimize this expression.  Thus, in total we have:

$$f(n,h) =\begin{cases} 0 \ &
  \mathrm{if}\ h = 0 \\
\infty \ & \mathrm{if}\ n=0\ \mathrm{and}\ h>0\\
h \ & \mathrm{if}\ n=1\\
1 + \min_{x=1,\ldots,h}\left\{\max\{f(n-1,x-1),f(n,h-x)\}\right\}
\ &  \mathrm{otherwise} \end{cases} .
$$

We can implement this easily using recursion:

\begin{verbatim}
def f(n, h):
    if h == 0:
        return 0
    elif n == 0:
        return float(`infinity')
    elif n == 1:
        return h
    else:
        ans = float(`infinity')
        for x in xrange(1, h+1):
            ans = min(ans, 1 + max(f(n-1, x-1), f(n, h-x)))
        return ans
\end{verbatim}

As usual, the above can be sped up using memoization.

\begin{verbatim}
def recurse(n, h, mem):
    if h == 0:
        return 0
    elif n == 0:
        return float(`infinity')
    elif n == 1:
        return h
    elif mem[n][h] != -1:
        return mem[n][h]
    else:
        ans = float(`infinity')
        for x in xrange(1, h+1):
            ans = min(ans, 1 + max(f(n-1, x-1), f(n, h-x)))
        mem[n][h] = ans
        return ans

def f(n, h):
    mem = []
    for i in xrange(n+1):
        mem += [[-1]*(h+1)]
    return recurse(n, h, mem)
\end{verbatim}

The running time of the memoized version is $O(nh^2)$ ($nh$
possibilities for the input, and $O(h)$ work in the \texttt{for} loop
for each input if not already in memory).  In fact it's even
$\Theta(nh^2)$.


\end{document}