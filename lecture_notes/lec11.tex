\documentclass[11pt]{article}

\usepackage{url,amsmath,amsthm,amsfonts,amssymb}
\usepackage{pgf}

\def\ceil#1{\lceil #1 \rceil}
\def\floor#1{\lfloor #1 \rfloor}
\def\ang#1{\langle #1 \rangle}

\newtheorem{definition}{Definition}
\newtheorem{remark}{Remark}
\newtheorem{theorem}{Theorem}
\newtheorem{lemma}[theorem]{Lemma}
\newtheorem{corollary}[theorem]{Corollary}
\newtheorem{proposition}[theorem]{Proposition}
\newtheorem{claim}[theorem]{Claim}
\newtheorem{observation}{Observation}

\newcommand{\R}{{\mathbb R}}
\newcommand{\Var}{\hbox{Var}}
\newcommand{\Z}{{\mathbb Z}}
\newcommand{\Q}{{\mathbb Q}}
\newcommand{\C}{{\mathbb C}}
\newcommand{\N}{{\mathbb N}}

\newlength{\toppush}
\setlength{\toppush}{2\headheight}
\addtolength{\toppush}{\headsep}

\newcommand{\htitle}[2]{\noindent\vspace*{-\toppush}\newline\parbox{6.5in}
 {\large Addis Ababa University, Amist Kilo \hfill #1\newline
\hspace*{\fill}{\bf Algorithms and Programming for High Schoolers} \hspace*{\fill} \newline
\mbox{}\hrulefill\mbox{}}\vspace*{1ex}\mbox{}\newline
\begin{center}{\Large\bf #2}\end{center}}

\newcommand{\handout}[3]{\thispagestyle{empty}
 \markboth{ #2}{ #2}
 \pagestyle{myheadings}\htitle{\protect\ref{#1}}{#2}{#3}}

\setlength{\oddsidemargin}{0pt}
\setlength{\evensidemargin}{0pt}
\setlength{\textwidth}{6.5in}
\setlength{\topmargin}{0in}
\setlength{\textheight}{8.5in}

\newcommand{\eps}{\varepsilon}
\newcommand{\sq}[1]{\langle#1\rangle}

\begin{document}

\htitle{August 1, 2016}{Lecture 11}

\paragraph{\Large Memoization: more worked examples.}

\paragraph{Party Fun:}\ 

\smallskip

\noindent (Problem source: Swiss Olympiad in Informatics 2004, see
\texttt{http://www.spoj.pl/problems/PARTY/}).  In this problem there
are $n$ parties.  The $i$th party has some entrance fee $c_i$ and
gives you some amount of fun $f_i$.  Given that you only have $D$
dollars, how can you choose which parties to attend so as to maximize
your fun?  This can be solved recursively.

\begin{verbatim}
# return the maximum fun that can be had from parties L[at:] given
# that we only have D dollars
def recurse(L, D, at):
    if at == len(L):
        return 0
    # don't attend party L[at]
    ans = recurse(L, D, at + 1)
    if D >= L[at][0]:
        # if we have enough money, try attending party L[at]
        ans = max(ans, L[at][1] + recurse(L, D - L[at][0], at + 1))
    return ans

# L is a list [[c_0,f_0], [c_1,f_1], ..., [c_{n-1}, f_{n-1}]]
def maximizeFun(L, D):
    return recurse(L, D, 0)
\end{verbatim}

The running time of the above implementation is $\Theta(2^n)$.  This
can be seen by drawing the recursion tree: if we have enough dollars
D to start off with, then at each level we branch in two ways (either
attend the party or don't).

We can obtain running time $\Theta(nD)$ by using memoization.

\begin{verbatim}
def recurse(L, D, at, mem):
    if at == len(L):
        return 0
    elif mem[at][D] != -1:
        return mem[at][D]
    ans = recurse(L, D, at + 1, mem)
    if D >= L[at][0]:
        ans = max(ans, L[at][1] + recurse(L, D - L[at][0], at + 1, mem))
    mem[at][D] = ans
    return ans

# we assume all fun amounts are positive; we wouldn't bother attending
# a party that doesn't have positive fun
def maximizeFun(L, D):
    mem = []
    for i in xrange(len(L)):
        mem += [[-1]*(D+1)]
    return recurse(L, D, 0, mem)
\end{verbatim}

\paragraph{Parenthesizing expressions:}\ 

\smallskip

\noindent (Problem source: Ulm Algorithm Course SoSe 2005, see
\texttt{http://www.spoj.pl/problems/LISA/}). You are given an
arithmetic expression with digits separated by `*' and `+'.  How can
you parenthesize the expression so as to maximize its value?  For
example, with the expression
$$ 1+2*3+4*5 ,$$
the best way of parenthesizing it is $(1+2)*(3+4)*5$, giving $105$.
For example, parenthesizing it as $1+(2*3)+(4*5)$ would only give
$27$.

Suppose the length of the expression is $n$.  We will show how to
solve the task in time $\Theta(n^3)$ using recursion and memoization.
First, a recursive solution:

\newpage

\begin{verbatim}
def maxValue(expr):
    if len(expr) == 1:
        return int(expr)
    else:
        ans = 0
        for op in xrange(1, len(expr), 2):
            # this loops over indices 1, 3, 5, ...
            # the "2" tells it to step by 2 each time
            if expr[op] == `+':
                ans = max(ans, maxValue(expr[:op]) + maxValue(expr[op+1:]))
            else:
                ans = max(ans, maxValue(expr[:op]) * maxValue(expr[op+1:]))
        return ans
\end{verbatim}

Now let's speed it up with memoization.

\begin{verbatim}
# return the max value that can be obtained by parenthesizing the
# expression expr[a:b+1]
def f(expr, a, b, mem):
    if a == b:
        return int(expr[a])
    elif mem[a][b] != -1:
        return mem[a][b]
    else:
        ans = 0
        for op in xrange(a + 1, b + 1, 2):
            if expr[op] == `+':
                ans = max(ans, f(expr(a, op-1, mem)) + f(expr, op+1, b, mem))
            else:
                ans = max(ans, f(expr(a, op-1, mem)) * f(expr, op+1, b, mem))
        mem[a][b] = ans
        return ans

def maxValue(expr):
    mem = []
    for i in xrange(len(expr)):
        mem += [[-1]*len(expr)]
    return f(expr, 0, len(expr)-1, mem)
\end{verbatim}

There are $n$ possibilities for each of $a,b$, and for each
possibility we do a while look taking $O(n)$ time.  The running time
is thus $O(n^3)$ (and in fact it is $\Theta(n^3)$).

\paragraph{Min cost bitonic tour:}
We have a bunch of points in the plane:
[[$x_0,y_0$],$\ldots$,[$x_{n-1},y_{n-1}$]].  We would like to start at
the $0$th point, visit all other points exactly once, then return to
the $0$th point.  We would like to do this while spending the least
amount of time traveling as possible with one constraint: we have to
do our travels west to east, then head back west again.  That is, we
can't zig-zag (we can't go east, then west, then back east again, then
back west again).  $x_0$ is the smallest amongst all the $x_i$, so the
$0$th point is the westernmost point.  Also, $x_{n-1}$ is the largest
amongst the $x_i$, so it is the easternmost point.  This is known as
the minimum cost bitonic tour problem.

We can solve this problem using dynamic programming.  The optimal path
goes east, possibly skipping some points along
the way, lands at $x_{n-1}$, then heads back west again, finally
ending back at $x_0$.  Let's instead think about our optimal path just
going east twice, by thinking about the return voyage in reverse.  So,
the first eastward path goes $x_0\rightarrow x_{i_1} \rightarrow
 \ldots \rightarrow x_{i_{r_1}} \rightarrow x_{n-1}$.  The return voyage,
in reverse, goes $x_0\rightarrow x_{j_1}\rightarrow \ldots \rightarrow
x_{j_{r_2}} \rightarrow x_{n-1}$.  We should not have any repeats
amongst the $x_i$ and $x_j$, and also we should have $r_1 + r_2 = n-2$
(we should visit all the points).

\paragraph{Example solution:}
\begin{verbatim}
import math

def distance(x, y):
    dx = x[0] - y[0]
    dy = x[1] - y[1]
    return math.sqrt(dx*dx + dy*dy)

def recurse(L, at1, at2, mem):
    if at1+1==len(L) and at2+1==len(L):
        return 0
    elif mem[at1][at2] != -1:
        return mem[at1][at]
    ans = float(`infinity')
    if at2 <= at1 + 1:
        for i in xrange(at2 + 1, len(L)):
            ans = min(ans, distance(L[at1], L[i]) + recurse(L, at2, i, mem))
    else:
        ans = recurse(L, at1 + 1, at2, mem)
    mem[at1][at2] = ans
    return ans
    
# We assume the points are already sorted from west to east, i.e. by
# their x coordinate. We also assume all x coordinates are unique.
def bitonicTour(L):
   mem = makeArray([len(A), len(A)], -1)
   return recurse(L, 0, 0, mem)
\end{verbatim}

The running time is $\Theta(n^2)$.  
There are $\Theta(n^2)$ possibilities for $a,b$, and only for $O(n)$ of these possibilities do we actually have to spend $\Theta(n)$ time; the rest only require $\Theta(1)$ time.
The basic idea of the solution is
to build both paths simultaneously from left to right.  Initially both
parts start at the $0$th point.  Then at any given stage, suppose one
path ends at L[at1] so far, and the other ends at L[at2], with
$\mathrm{at1}\le\mathrm{at2}$.  Since we have to visit all points, if
there are points between L[at1] and L[at2], then we have no choice:
the one path has to visit L[at1+1] next.  Otherwise, if there are no
points in between (i.e.\ at2 and at1 differ by at most $1$), then the
path that ended at L[at1] can choose where to go next, to some point
after L[at2], possiby skipping some number of points.  We try all
possibilities and take the best choice.


\end{document}