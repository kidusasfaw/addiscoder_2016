%%
%% Style file "borrowed" from Michel Goemans
%%
\documentclass[11pt]{article}

\usepackage{url,amsmath,amsthm,amsfonts,amssymb}

\def\ceil#1{\lceil #1 \rceil}
\def\floor#1{\lfloor #1 \rfloor}
\def\ang#1{\langle #1 \rangle}

\newtheorem{definition}{Definition}
\newtheorem{remark}{Remark}
\newtheorem{theorem}{Theorem}
\newtheorem{lemma}[theorem]{Lemma}
\newtheorem{corollary}[theorem]{Corollary}
\newtheorem{proposition}[theorem]{Proposition}
\newtheorem{claim}[theorem]{Claim}
\newtheorem{observation}{Observation}

\newcommand{\R}{{\mathbb R}}
\newcommand{\Var}{\hbox{Var}}
\newcommand{\Z}{{\mathbb Z}}
\newcommand{\Q}{{\mathbb Q}}
\newcommand{\C}{{\mathbb C}}
\newcommand{\N}{{\mathbb N}}

\newlength{\toppush}
\setlength{\toppush}{2\headheight}
\addtolength{\toppush}{\headsep}

\newcommand{\htitle}[2]{\noindent\vspace*{-\toppush}\newline\parbox{6.5in}
 {\large Addis Ababa University, Amist Kilo \hfill #1\newline
\hspace*{\fill}{\bf Algorithms and Programming for High Schoolers} \hspace*{\fill} \newline
\mbox{}\hrulefill\mbox{}}\vspace*{1ex}\mbox{}\newline
\begin{center}{\Large\bf #2}\end{center}}

\newcommand{\handout}[3]{\thispagestyle{empty}
 \markboth{ #2}{ #2}
 \pagestyle{myheadings}\htitle{\protect\ref{#1}}{#2}{#3}}

\setlength{\oddsidemargin}{0pt}
\setlength{\evensidemargin}{0pt}
\setlength{\textwidth}{6.5in}
\setlength{\topmargin}{0in}
\setlength{\textheight}{8.5in}

\newcommand{\eps}{\varepsilon}
\newcommand{\sq}[1]{\langle#1\rangle}

\begin{document}

\htitle{July 21, 2016}{Lecture 4}

\paragraph{\Large Recursion:}

{\em Recursion} in computer science solving a problem by solving
simpler instantiations of the same problem.  As a classic example,
consider the {\em Fibonacci} sequence $1,1,3,5,8,13,\ldots$.  This
sequence is defined by the $0$th and $1$st Fibonacci numbers both
being $1$, and subsequent Fibonacci numbers being the sum of the
previous two.

That is, if $F_i$ represents the $i$th Fibonacci number,
$$
F_i = \begin{cases} 1 \ &
  \mathrm{if}\ i=0\ \mathrm{or}\ i = 1
\\ F_{i-1} + F_{i-2} \
&\mathrm{otherwise} \end{cases}
$$

Now here is an example of using recursion to calculate the $n$th
Fibonacci number.  Note that the \texttt{fibonacci} function calls
itself on smaller, i.e.\ simpler, inputs.

\begin{verbatim}
def fibonacci(n):
    if n<2:
        return 1
    return fibonacci(n-1) + fibonacci(n-2)
\end{verbatim}

Let's give another example.  Suppose our friend Bob has a number
between $1$ and $100$ and isn't telling us what it is.  However, he's
implemented a \texttt{bool} function \texttt{isGreaterThan}(n) for us
which returns \texttt{True} if his number is greater than n and
\texttt{False} otherwise.  Now, let's implement a function
\texttt{searchForBobsNumber}().

\begin{verbatim}
def searchForBobsNumber():
    x = 1
    while isGreaterThan(x):
        x += 1
    return x
\end{verbatim}

We could also implement this function using a recursive helper
function.

\begin{verbatim}
# We assume the answer is in the range [a,b]
def searchForBobsNumberHelper(a, b):
    if isGreaterThan(a):
        return searchForBobsNumberHelper(a+1, b)
    else:
        return a

def searchForBobsNumber():
    return searchForBobsNumber(1, 100)
\end{verbatim}

\paragraph{\Large Binary Search:}

If Bob's number is $100$, potentially the \texttt{while} loop in the
above implementation would run for $100$ steps before finding Bob's
number.  The implementation would be even slower if Bob's number could
be in the range from $1$ to one billion, and unbearably slow if in the
range from $1$ to one trillion.  One way of remedying this is
using a technique known as {\em binary search}.  Suppose Bob's number
is between $1$ and $m$ and we first ask whether his number is greater
than $\floor{m/2}$.  Then, no matter what his number is, we will
eliminate roughly half of the possibilities, and we can recursively
search in the range that is left.  
That is, consider the following
faster version of \texttt{searchForBobsNumberHelper(a,b)} (remember
that \texttt{int} division in Python rounds down to the nearest
\texttt{int}):

\begin{verbatim}
# returns True if n equals Bob's number, and False otherwise
def isEqualTo(n):
    return isGreaterThan(n-1) and not isGreaterThan(n)

# We assume the answer is in the range [a,b]
def searchForBobsNumberHelper(a, b):
    mid = (a+b)/2
    if isEqualTo(mid):
        return mid
    elif isGreaterThan(mid):
        return searchForBobsNumberHelper(mid+1, b)
    else:
        return searchForBobsNumberHelper(a, mid)
\end{verbatim}

Suppose Bob's number can be in the range $[1,m]$, and for simplicity,
let's assume $m$ is a power of $2$ (if it isn't, we'll just pretend
his number can be in the range $[1,q]$ where $q$ is the smallest power
of $2$ greater than or equal to $m$).  Since we cut down
the number of possibilities in half each time, and
we will get the answer right away if there is only one possibility
left, the number of times we actually need to ask Bob if his number is
greater is the smallest number $k$ such that $m/2^k = 1$, which gives
$k = \log_2 m$.  

Compare this with the \texttt{while} loop
implementation in the previous section, which could take $m$
steps. This is a big difference for large $m$; note that the $\log$
base $2$ of one trillion is only about $40$.  Many of you are probably
familiar with the term ``Gigahertz'' corresponding to the speed of the
CPU in your computer.  This, roughly, corresponds to the number of
instructions your computer can execute in one second, with 1GHz
meaning one billion instructions per second (this isn't {\em quite}
what it means, but it's a good enough approximation for this
discussion).  Thus, the
\texttt{while} loop implementation for $m$ being one trillion on a
1GHz machine would
take at least $1000$ seconds, which is a bit over $16$ minutes. In
fact, it
would take even more time since some instructions take longer, and
some steps which are just one line in Python actually correspond to
multiple
instructions when you translate to machine language.  Meanwhile, the
binary search implementation would execute almost
instantaneously. From now on in the class, this will be our focus when
studying {\em algorithms}: finding ways of implementing functions
such that they run quickly.  Note that the \texttt{while} loop and
binary search implementations in this section both give the right
answer, but the latter is vastly superior when it comes to running
time.

\end{document}
