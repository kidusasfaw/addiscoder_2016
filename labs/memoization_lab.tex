%%
%% Style file "borrowed" from Michel Goemans
%%
\documentclass[11pt]{article}

\usepackage{url,amsmath,amsthm,amsfonts,amssymb}
\usepackage{pgf}

\def\ceil#1{\lceil #1 \rceil}
\def\floor#1{\lfloor #1 \rfloor}
\def\ang#1{\langle #1 \rangle}

\newtheorem{definition}{Definition}
\newtheorem{remark}{Remark}
\newtheorem{theorem}{Theorem}
\newtheorem{lemma}[theorem]{Lemma}
\newtheorem{corollary}[theorem]{Corollary}
\newtheorem{proposition}[theorem]{Proposition}
\newtheorem{claim}[theorem]{Claim}
\newtheorem{observation}{Observation}

\newcommand{\R}{{\mathbb R}}
\newcommand{\Var}{\hbox{Var}}
\newcommand{\Z}{{\mathbb Z}}
\newcommand{\Q}{{\mathbb Q}}
\newcommand{\C}{{\mathbb C}}
\newcommand{\N}{{\mathbb N}}

\newlength{\toppush}
\setlength{\toppush}{2\headheight}
\addtolength{\toppush}{\headsep}

\newcommand{\htitle}[2]{\noindent\vspace*{-\toppush}\newline\parbox{6.5in}
 {\large Addis Ababa University, Amist Kilo \hfill #1\newline
\hspace*{\fill}{\bf Algorithms and Programming for High Schoolers} \hspace*{\fill} \newline
\mbox{}\hrulefill\mbox{}}\vspace*{1ex}\mbox{}\newline
\begin{center}{\Large\bf #2}\end{center}}

\newcommand{\handout}[3]{\thispagestyle{empty}
 \markboth{ #2}{ #2}
 \pagestyle{myheadings}\htitle{\protect\ref{#1}}{#2}{#3}}

\setlength{\oddsidemargin}{0pt}
\setlength{\evensidemargin}{0pt}
\setlength{\textwidth}{6.5in}
\setlength{\topmargin}{0in}
\setlength{\textheight}{8.5in}

\newcommand{\eps}{\varepsilon}
\newcommand{\sq}[1]{\langle#1\rangle}

\begin{document}

\htitle{July 20, 2011}{Lab 12}

\paragraph{Exercise 1:}
Modify the code for the Bellman-Ford single-source shortest path
algorithm so that it takes as input the origin $x$ and destination $y$
and returns the actual shortest {\em path} from $x$ to $y$ and not
just the length of the shortest path.  You can assume that $y$ is
reachable from $x$ and that there are no negative weight cycles.

\paragraph{Exercise 2:}
Modify the code for Floyd-Warshall (either the recursive or iterative
approach) to keep track of a matrix \texttt{next}[][] so that
\texttt{next}[u][v] is some intermediate vertex on the shortest path
from u to v (or the Python value \texttt{None} if there is no
intermediate vertex).  Now write a recursive procedure
\texttt{findPath} which
takes $u$, $v$, and the \texttt{dist} and \texttt{next} matrices, and
returns the shortest path from $u$ to $v$ as a list.  For example, if
the shortest path from $0$ to $3$ is $0\rightarrow 1\rightarrow 3$,
then \texttt{findPath} should return [0,1,3].

\paragraph{Exercise 3:}
Modify the code for Floyd-Warshall to return -1 if there exists a
negative weight cycle somewhere in the graph.  Otherwise, it should
return a matrix of distances as before.

\end{document}
