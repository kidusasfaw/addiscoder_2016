%%
%% Style file "borrowed" from Michel Goemans
%%
\documentclass[11pt]{article}

\usepackage{url,amsmath,amsthm,amsfonts,amssymb}
\usepackage[usenames]{color}
\newcommand{\ans}[1]{\textcolor{red}{#1}}

\def\ceil#1{\lceil #1 \rceil}
\def\floor#1{\lfloor #1 \rfloor}
\def\ang#1{\langle #1 \rangle}

\newtheorem{definition}{Definition}
\newtheorem{remark}{Remark}
\newtheorem{theorem}{Theorem}
\newtheorem{lemma}[theorem]{Lemma}
\newtheorem{corollary}[theorem]{Corollary}
\newtheorem{proposition}[theorem]{Proposition}
\newtheorem{claim}[theorem]{Claim}
\newtheorem{observation}{Observation}

\newcommand{\R}{{\mathbb R}}
\newcommand{\Var}{\hbox{Var}}
\newcommand{\Z}{{\mathbb Z}}
\newcommand{\Q}{{\mathbb Q}}
\newcommand{\C}{{\mathbb C}}
\newcommand{\N}{{\mathbb N}}

\newlength{\toppush}
\setlength{\toppush}{2\headheight}
\addtolength{\toppush}{\headsep}

\newcommand{\htitle}[2]{\noindent\vspace*{-\toppush}\newline\parbox{6.5in}
 {\large Sports Academy \hfill #1\newline
\hspace*{\fill}{\bf Algorithms and Programming for High Schoolers} \hspace*{\fill} \newline
\mbox{}\hrulefill\mbox{}}\vspace*{1ex}\mbox{}\newline
\begin{center}{\Large\bf #2}\end{center}}

\newcommand{\handout}[3]{\thispagestyle{empty}
 \markboth{ #2}{ #2}
 \pagestyle{myheadings}\htitle{\protect\ref{#1}}{#2}{#3}}

\setlength{\oddsidemargin}{0pt}
\setlength{\evensidemargin}{0pt}
\setlength{\textwidth}{6.5in}
\setlength{\topmargin}{0in}
\setlength{\textheight}{8.5in}

\newcommand{\eps}{\varepsilon}
\newcommand{\sq}[1]{\langle#1\rangle}

\begin{document}

\htitle{2016}{Lab 3 Solutions}

\paragraph{Exercise 1:}
Consider the {\em Trionacci} sequence defined as follows.
$$
T_i = \begin{cases} 1 \ &
  \mathrm{if}\ i=0\ \mathrm{or}\ i = 1 \ \mathrm{or}\ i=2
\\ T_{i-1} + T_{i-2} + T_{i-3} \
&\mathrm{otherwise} \end{cases}
$$

Implement a function \texttt{trionacci}(n) which returns the $n$th
Trionacci number.

\paragraph{Example solution:}
\begin{verbatim}
def trionacci(n):
    if n<3:
        return 1
    else:
        return trionacci(n-1) + trionacci(n-2) + trionacci(n-3)
\end{verbatim}

\paragraph{Exercise 2:}
The {\em factorial} of $n$ is $n! = 1\cdot 2 \cdot \ldots \cdot n$ (we
define $0! = 1$).
Implement \texttt{factorial}(n) in two ways: one using a
\texttt{while} loop, and the other using recursion.

\paragraph{Example solution:}
\begin{verbatim}
# using a while loop
def factorial(n):
    ans = 1
    x = 1
    while x <= n:
        ans *= x
        x += 1
    return ans
\end{verbatim}

\begin{verbatim}
# using recursion
def factorial(n):
    if n==0:
        return 1
    else:
        return n*factorial(n-1)
\end{verbatim}

\paragraph{Exercise 3:}
Last lab we had the following exercise:

\medskip

An integer is said to be a {\em palindrome} if its digits are the same
forward and backwards (not including leading zeroes). For example,
12321 is a palindrome, as is 5.  1231 on the other hand is not a
palindrome, and neither is 50 (remember we are not including leading
zeroes).
Write a function \texttt{isPalindrome}(n) which returns
\texttt{True} if n is a palindrome and \texttt{False} otherwise.

\medskip

In today's lab, implement \texttt{isPalindrome} using recursion.
Specifically, check if the first and last characters are equal, and
recurse on the middle substring if required.

\paragraph{Example solution:}
\begin{verbatim}
def isPalindrome(s):
    if len(s) < 2:
        return True
    return s[0]==s[len(s)-1] and isPalindrome(s[1:len(s)-1])
\end{verbatim}

\paragraph{Exercise 4:}
Define a function \texttt{flooredSquareRoot}(n)
which takes a positive \texttt{int} or \texttt{long} n and computes
its square root, rounded down to the nearest integer.  Python has a
buit-in \texttt{sqrt} function which could be helpful here, but don't
use it.

Do two implementations.  In the first, use a while loop starting from
$0$ and going upward.  Call that function
\texttt{slowFlooredSquareRoot}(n). Next, implement
\texttt{flooredSquareRoot}(n) using binary search.  Experiment by
evaluating these functions on various inputs.  Try n being a billion
--- notice a difference in the time it takes to compute the answer?

\paragraph{Example solution:}
\begin{verbatim}
def slowFlooredSquareRoot(n):
    x = 0
    while x*x <= n:
        x += 1
    return x-1
\end{verbatim}

\begin{verbatim}
# we assume the floored square root is in the interval [a,b]
def flooredSquareRootHelper(a, b, n):
    mid = (a+b+1)/2
    if a==b:
        return a
    elif mid*mid == n:
        return mid
    elif mid*mid < n:
        return flooredSquareRootHelper(mid, b, n)
    else:
        return flooredSquareRootHelper(a, mid-1, n)

def flooredSquareRoot(n):
    return flooredSquareRootHelper(0, n, n)
\end{verbatim}

\paragraph{Exercise 5:}
Implement a function \texttt{calcNthSmallest}(n, intervals) which
takes as
input a nonnegative \texttt{int} n, and a list of intervals
$[[a_1,b_1],\ldots,[a_m,b_m]]$ and
calculates the $n$th smallest number (0-indexed) when taking the union
of all the intervals with repetition.  For example, if the intervals
were $[1,5],[2,4],[7,9]$, their union with repetition would be
$\{1,2,2,3,3,4,4,5,7,8,9\}$ (note $2,3,4$ each appear twice since
they're in both the intervals $[1,5]$ and $[2,4]$).  For this list of
intervals, the $0$th smallest number would be $1$, and the $3$rd and
$4$th smallest would both be $3$.

Your implementation should run quickly even when the $a_i,b_i$ can be
very large (like, one trillion), and there are several
intervals. 

\paragraph{Example solution:}
First, here are some helper functions which will be useful.
\begin{verbatim}
# compute the index of the first time x appears in the union of intervals
def firstTime(x, intervals):
    answer = 0
    for L in intervals:
        if x > L[1]:
            answer += L[1] - L[0] + 1
        elif x > L[0]:
            answer += x - L[0]
    return answer

# compute the index of the last time x appears in the union of intervals
def lastTime(x, intervals):
    answer = 0
    for L in intervals:
        if x >= L[1]:
            answer += L[1] - L[0] + 1
        elif x >= L[0]:
            answer += x - L[0] + 1
    return answer-1
\end{verbatim}

Now, here is a slow implementation of \texttt{calcNthSmallest}(n,
intervals) (at least, it is slow when the intervals can be very long).

\begin{verbatim}
def calcNthSmallest(n, intervals):
    for L in intervals:
        for x in xrange(L[0], L[1]+1):
            first = firstTime(x, intervals)
            last = lastTime(x, intervals)
            if first<=n and n<=last:
                return x
\end{verbatim}

The reason it is slow for long intervals is that we loop over the
entire range
from L[0] to L[1]+1.  To make this faster, we can use a binary
search over the interval [L[0], L[1]].

\begin{verbatim}
# Binary searches for the nth smallest number being in the interval
# [a,b]. If no such number in [a,b] is found, [False, `'] is returned.
# Otherwise, [True, x] is returned, where x is the nth smallest
# number. 
def binarySearch(a, b, n, intervals):
    if a>b:
        return [False, `']
    mid = (a+b)/2
    first = firstTime(mid, intervals)
    last = lastTime(mid, intervals)
    if first<=n and n<=last:
        return [True, mid]
    elif first>n:
        return binarySearch(a, mid-1, n, intervals)
    else:
        return binarySearch(mid+1, b, n, intervals)

def calcNthSmallest(n, intervals):
    # The answer has to be in one of the intervals, so try them all in
      a for loop.
    for L in intervals:
        answer = binarySearch(L[0], L[1], n, intervals)
        if answer[0]:
            return answer[1]
\end{verbatim}

\end{document}