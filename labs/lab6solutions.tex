%%
%% Style file "borrowed" from Michel Goemans
%%
\documentclass[11pt]{article}

\usepackage{url,amsmath,amsthm,amsfonts,amssymb}
\usepackage[usenames]{color}
\newcommand{\ans}[1]{\textcolor{red}{#1}}

\def\ceil#1{\lceil #1 \rceil}
\def\floor#1{\lfloor #1 \rfloor}
\def\ang#1{\langle #1 \rangle}

\newtheorem{definition}{Definition}
\newtheorem{remark}{Remark}
\newtheorem{theorem}{Theorem}
\newtheorem{lemma}[theorem]{Lemma}
\newtheorem{corollary}[theorem]{Corollary}
\newtheorem{proposition}[theorem]{Proposition}
\newtheorem{claim}[theorem]{Claim}
\newtheorem{observation}{Observation}

\newcommand{\R}{{\mathbb R}}
\newcommand{\Var}{\hbox{Var}}
\newcommand{\Z}{{\mathbb Z}}
\newcommand{\Q}{{\mathbb Q}}
\newcommand{\C}{{\mathbb C}}
\newcommand{\N}{{\mathbb N}}

\newlength{\toppush}
\setlength{\toppush}{2\headheight}
\addtolength{\toppush}{\headsep}

\newcommand{\htitle}[2]{\noindent\vspace*{-\toppush}\newline\parbox{6.5in}
 {\large Sports Academy \hfill #1\newline
\hspace*{\fill}{\bf Algorithms and Programming for High Schoolers} \hspace*{\fill} \newline
\mbox{}\hrulefill\mbox{}}\vspace*{1ex}\mbox{}\newline
\begin{center}{\Large\bf #2}\end{center}}

\newcommand{\handout}[3]{\thispagestyle{empty}
 \markboth{ #2}{ #2}
 \pagestyle{myheadings}\htitle{\protect\ref{#1}}{#2}{#3}}

\setlength{\oddsidemargin}{0pt}
\setlength{\evensidemargin}{0pt}
\setlength{\textwidth}{6.5in}
\setlength{\topmargin}{0in}
\setlength{\textheight}{8.5in}

\newcommand{\eps}{\varepsilon}
\newcommand{\sq}[1]{\langle#1\rangle}

\begin{document}

\htitle{2016}{Lab 6 Solutions}

\paragraph{Exercise 1:}
Write a recursive procedure \texttt{addDigits}(n) which takes a
nonnegative integer n and returns the sum of the digits of n.

\paragraph{Example solution:}
\begin{verbatim}
def addDigits(n):
    if n==0:
        return 0
    return (n%10) + addDigits(n/10)
\end{verbatim}

\paragraph{Exercise 2:}
Python already has a function \texttt{reverse()} for lists
(\texttt{L.reverse()} reverses the \texttt{list} L).  Let's implement
our own \texttt{reverse()} function using recursion.
\texttt{reverse}(L) should be a recursive function that outputs a
\texttt{list} which contains the elements of L in reverse. Remember
that L[i:] evaluates to a \texttt{list} containing only the elements
of L from index i onward.

\paragraph{Example solution:}
\begin{verbatim}
def reverse(L):
    if len(L)==0:
        return []
    return [L[len(L)-1]] + reverse(L[:len(L)-1])
\end{verbatim}

\paragraph{Exercise 3:}
Write a recursive procedure \texttt{minElement}(L) which takes a
\texttt{list}  L of integers and returns the minimum element in the
\texttt{list}.

\paragraph{Example solution:}
\begin{verbatim}
def minElement(L):
    if len(L)==1:
        return L[0]
    return min(L[0], minElement(L[1:]))
\end{verbatim}

\paragraph{Exercise 4:}
A {\em superknight} is on a chessboard, at grid location $(0,0)$ (the
bottom left corner).  How many ways can he get to the location
$(x,y)$ if his allowed moves are given in the \texttt{list} L?  Write
a function \texttt{numKnightWays}(x,y,L) that returns this number.
Each element in L is a list of size two [i,j] signifying that it is
possible for the knight to move from $(a,b)$ to $(a+i, b+j)$.  $i,j$
are always both positive.

\paragraph{Example solution:}
\begin{verbatim}
# return the number of ways to get to x,y given that we are currently
# at position (atx, aty)
def knightRecurse(atx, aty, x, y, L):
    if (atx>x) or (aty>y):
        return 0
    elif atx==x and aty==y:
        return 1
    ans = 0
    for t in L:
        ans += knightRecurse(atx+t[0], aty+t[1], x, y, L)
    return ans

def numKnightWays(x, y, L):
    return knightRecurse(0, 0, x, y, L)
\end{verbatim}

The above solution can be sped up using memoization, by memoizing
based on \texttt{atx} and \texttt{aty}.

\begin{verbatim}
def knightRecurse(atx, aty, x, y, L, mem):
    if (atx>x) or (aty>y):
        return 0
    elif atx==x and aty==y:
        return 1
    elif mem[atx][aty] != -1:
        return mem[atx][aty]
    mem[atx][aty] = 0
    for t in L:
        mem[atx][aty] += knightRecurse(atx+t[0], aty+t[1], x, y, L, mem)
    return mem[atx][aty]

def numKnightWays(x, y, L):
    mem = []
    for i in xrange(x+1):
        mem += [[-1]*(y+1)]
    return knightRecurse(0, 0, x, y, L, mem)
\end{verbatim}

\paragraph{Exercise 5:}
An {\em expression} is defined recursively as follows.  An integer is
an expression, which evaluates to the integer itself.  If
\texttt{EXPR} is an expression, then so is
(\texttt{EXPR}), and it evaluates to whatever \texttt{EXPR} evaluated
to.  Finally, if \texttt{EXPR1} and \texttt{EXPR2} are
expressions, then (\texttt{OP} \texttt{EXPR1} \texttt{EXPR2}) is an
expression, where \texttt{OP} can be any one of $+,-,*$, and it
evaluates to \texttt{evaluate}(\texttt{EXPR1}) \texttt{OP}
\texttt{evaluate}(\texttt{EXPR2}).  You should
write a function \texttt{evaluate} which takes a \texttt{str} and
evaluates the expression it is a valid expression, and outputs
``INVALID'' if it is not a valid expression.  For example:

\begin{itemize}
\item \texttt{evaluate}(`(+ 1 5)') gives 6.
\item \texttt{evaluate}(`(* 3 (- 5 2))') gives 9 (first (- 5 2) is
  evaluated as $5-2 = 3$, and then we have $3*3 = 9$).
\item \texttt{evaluate}(`(+ 1 (+ 5))') gives ``INVALID'' since (+ 5)
  is not a valid expression.
\item \texttt{evaluate}(`()') gives ``INVALID'' since the empty string
  is not a valid expression.
\end{itemize}

\paragraph{Example solution:}
Below is a recursive solution.

\begin{verbatim}
# returns True if c represents a digit from `0' to `9'
# and False otherwise
def isDigit(c):
    return c>=`0' and c<=`9'

def isOperator(c):
    return c==`+' or c==`-' or c==`*'

# find the first prefix of expr which could be a valid expression
# return INVALID of no such prefix exists
def locateExpression(expr):
    if len(expr) == 0:
        return `INVALID'
    elif isDigit(expr[0]):
        # try to build an integer expression as a prefix
        at = 1
        while at<len(expr) and isDigit(expr[at]):
            at += 1
        return expr[:at]
    elif expr[0] == `-':
        # try to build a negative integer expression as a prefix
        if len(expr) == 1:
            return `INVALID'
        elif not isDigit(expr[1]):
            return `INVALID'
        at = 2
        while at<len(expr) and isDigit(expr[at]):
            at += 1
        return expr[:at]
    elif expr[0] == `(':
        # find the matching parenthesis
        x = 1
        at = 1
        while at<len(expr) and x>0:
            if expr[at]==`(':
                x += 1
            elif expr[at]==`)':
                x -= 1
            at += 1
        if x != 0:
            return `INVALID'
        return expr[:at]
    else:
        return `INVALID'

def evaluate(expr):
    if len(expr) == 0:
        return `INVALID'
    elif expr[0] != `(':
        if locateExpression(expr) == expr:
            return int(expr)
        else:
            return `INVALID'
    else:
        if len(expr) == 1:
            # if the first letter is `(', we at least need `)' at end
            return `INVALID'
        elif expr[len(expr)-1] != `)':
            return `INVALID'
        elif isOperator(expr[1]):
            # in this case we need to apply an OP to two exprs
            if (len(expr) < 7) or (expr[2] != ` '):
                # we need at least 7 characters for (, OP, two spaces, and two exprs
                return `INVALID'
            expr1 = locateExpression(expr[3:])
            if expr1 == `INVALID':
                return `INVALID'
            elif len(expr1) >= len(expr)-4:
                # if expr1 is too long, there's no room left for expr2
                return `INVALID'
            elif expr[3+len(expr1)] != ` ':
                # a space should separate the two expressions
                return `INVALID'
            expr2 = locateExpression(expr[4+len(expr1):])
            if expr2 == `INVALID':
                return `INVALID'
            elif len(expr1)+len(expr2)+5 != len(expr):
                # expr should be (OP space expr1 space expr2)
                return `INVALID'
            # evaluate the two expressions recursively
            A = evaluate(expr1)
            B = evaluate(expr2)
            if (A==`INVALID') or (B==`INVALID'):
                return `INVALID'
            elif expr[1] == `+':
                return A + B
            elif expr[1] == `-':
                return A - B
            else:
                return A * B
        else:
            return evaluate(expr[1:len(expr)-1])
\end{verbatim}

\end{document}