%%
%% Style file "borrowed" from Michel Goemans
%%
\documentclass[11pt]{article}

\usepackage{url,amsmath,amsthm,amsfonts,amssymb}
\usepackage{pgf}

\def\ceil#1{\lceil #1 \rceil}
\def\floor#1{\lfloor #1 \rfloor}
\def\ang#1{\langle #1 \rangle}

\newtheorem{definition}{Definition}
\newtheorem{remark}{Remark}
\newtheorem{theorem}{Theorem}
\newtheorem{lemma}[theorem]{Lemma}
\newtheorem{corollary}[theorem]{Corollary}
\newtheorem{proposition}[theorem]{Proposition}
\newtheorem{claim}[theorem]{Claim}
\newtheorem{observation}{Observation}

\newcommand{\R}{{\mathbb R}}
\newcommand{\Var}{\hbox{Var}}
\newcommand{\Z}{{\mathbb Z}}
\newcommand{\Q}{{\mathbb Q}}
\newcommand{\C}{{\mathbb C}}
\newcommand{\N}{{\mathbb N}}

\newlength{\toppush}
\setlength{\toppush}{2\headheight}
\addtolength{\toppush}{\headsep}

\newcommand{\htitle}[2]{\noindent\vspace*{-\toppush}\newline\parbox{6.5in}
 {\large Sports Academy \hfill #1\newline
\hspace*{\fill}{\bf Algorithms and Programming for High Schoolers} \hspace*{\fill} \newline
\mbox{}\hrulefill\mbox{}}\vspace*{1ex}\mbox{}\newline
\begin{center}{\Large\bf #2}\end{center}}

\newcommand{\handout}[3]{\thispagestyle{empty}
 \markboth{ #2}{ #2}
 \pagestyle{myheadings}\htitle{\protect\ref{#1}}{#2}{#3}}

\setlength{\oddsidemargin}{0pt}
\setlength{\evensidemargin}{0pt}
\setlength{\textwidth}{6.5in}
\setlength{\topmargin}{0in}
\setlength{\textheight}{8.5in}

\newcommand{\eps}{\varepsilon}
\newcommand{\sq}[1]{\langle#1\rangle}

\begin{document}

\htitle{July 20, 2011}{Lab 12}

\paragraph{Exercise 1:}
Modify the code for the Bellman-Ford single-source shortest path
algorithm so that it takes as input the origin $x$ and destination $y$
and returns the actual shortest {\em path} from $x$ to $y$ and not
just the length of the shortest path.  You can assume that $y$ is
reachable from $x$ and that there are no negative weight cycles.

\paragraph{Example solution:}
The basic idea is to add a list \texttt{from} of length $n$, where $n$
is the number of vertices.  \texttt{from}[u] is the vertex right
before u on the shortest path from x to u (if u is x, then we set
from[u] to $-1$).

\begin{verbatim}
def bellmanFord(A, x, y):
    # E is a list of edges with weights
    E = []
    for i in xrange(len(A)):
        for p in A[i]:
            E += [[i] + p]
    # dist[i] is the length of the shortest path to i
    dist = [float(`infinity')]*len(A)
    from = [-1]*len(A)
    dist[x] = 0
    for i in xrange(len(A) - 1):
        for e in E:
            u = e[0]
            v = e[1]
            weight = e[2]
            if dist[u] + weight < dist[v]:
                from[v] = u
                dist[v] = weight

    # look for negative weight cycles
    for e in E:
        u = e[0]
        v = e[1]
        weight = e[2]
        if dist[u] + weight < dist[v]:
            return -1

    # build the shortest path, from the end to the beginning
    L = []
    at = y
    while at != -1:
        L += [at]
        at = from[at]
    L.reverse()
    return L 
\end{verbatim}

\paragraph{Exercise 2:}
Modify the code for Floyd-Warshall (either the recursive or iterative
approach) to keep track of a matrix \texttt{next}[][] so that
\texttt{next}[u][v] is some intermediate vertex on the shortest path
from u to v (or the Python value \texttt{None} if there is no
intermediate vertex).  Now write a recursive procedure
\texttt{findPath} which
takes $u$, $v$, and the \texttt{dist} and \texttt{next} matrices, and
returns the shortest path from $u$ to $v$ as a list.  For example, if
the shortest path from $0$ to $3$ is $0\rightarrow 1\rightarrow 3$,
then \texttt{findPath} should return [0,1,3].

\paragraph{Example solution:}
In fact, \texttt{findPath} does not need the \texttt{dist} matrix.

\begin{verbatim}
# findPath should be called with the next list that's generated in
# floyd Warshall
def findPath(u, v, next):
    # if there's no intermediate vertex, we just directly take the
    # edge (u,v)
    if next[u][v] == -1:
        return [u, v]
    # otherwise we recurse. in the second recursion we don't take the
    # 0th element so that next[u][v] doesn't show up twice in the
    # final output
    return findPath(u, next[u][v], next) + findPath(next[u][v], v, next)[1:]

def floydWarshall(w):
    # now dist is a copy of the weight function
    dist = w[:]
    next = []
    for i in xrange(len(w)):
        next += [[-1]*len(w)]

    for k in xrange(len(w)):
        for u in xrange(len(w)):
            for v in xrange(len(w)):
                if dist[u][k] + dist[k][v] < dist[u][v]:
                    dist[u][v] = dist[u][k] + dist[k][v]
                    next[u][v] = k
    return dist
\end{verbatim}

\paragraph{Exercise 3:}
Modify the code for Floyd-Warshall to return -1 if there exists a
negative weight cycle somewhere in the graph.  Otherwise, it should
return a matrix of distances as before.

\paragraph{Example solution:}
There's a negative weight cycle if and only if some vertex has a
negative distance to itself.

\begin{verbatim}
def floydWarshall(w):
    # now dist is a copy of the weight function
    dist = w[:]
    for k in xrange(len(w)):
        for u in xrange(len(w)):
            for v in xrange(len(w)):
                dist[u][v] = min(dist[u][v], dist[u][k] + dist[k][v])
    for i in xrange(len(w)):
        if dist[i][i] < 0:
            return -1
    return dist
\end{verbatim}

\end{document}
