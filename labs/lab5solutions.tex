%%
%% Style file "borrowed" from Michel Goemans
%%
\documentclass[11pt]{article}

\usepackage{url,amsmath,amsthm,amsfonts,amssymb}
\usepackage[usenames]{color}
\newcommand{\ans}[1]{\textcolor{red}{#1}}

\def\ceil#1{\lceil #1 \rceil}
\def\floor#1{\lfloor #1 \rfloor}
\def\ang#1{\langle #1 \rangle}

\newtheorem{definition}{Definition}
\newtheorem{remark}{Remark}
\newtheorem{theorem}{Theorem}
\newtheorem{lemma}[theorem]{Lemma}
\newtheorem{corollary}[theorem]{Corollary}
\newtheorem{proposition}[theorem]{Proposition}
\newtheorem{claim}[theorem]{Claim}
\newtheorem{observation}{Observation}

\newcommand{\R}{{\mathbb R}}
\newcommand{\Var}{\hbox{Var}}
\newcommand{\Z}{{\mathbb Z}}
\newcommand{\Q}{{\mathbb Q}}
\newcommand{\C}{{\mathbb C}}
\newcommand{\N}{{\mathbb N}}

\newlength{\toppush}
\setlength{\toppush}{2\headheight}
\addtolength{\toppush}{\headsep}

\newcommand{\htitle}[2]{\noindent\vspace*{-\toppush}\newline\parbox{6.5in}
 {\large Sports Academy \hfill #1\newline
\hspace*{\fill}{\bf Algorithms and Programming for High Schoolers} \hspace*{\fill} \newline
\mbox{}\hrulefill\mbox{}}\vspace*{1ex}\mbox{}\newline
\begin{center}{\Large\bf #2}\end{center}}

\newcommand{\handout}[3]{\thispagestyle{empty}
 \markboth{ #2}{ #2}
 \pagestyle{myheadings}\htitle{\protect\ref{#1}}{#2}{#3}}

\setlength{\oddsidemargin}{0pt}
\setlength{\evensidemargin}{0pt}
\setlength{\textwidth}{6.5in}
\setlength{\topmargin}{0in}
\setlength{\textheight}{8.5in}

\newcommand{\eps}{\varepsilon}
\newcommand{\sq}[1]{\langle#1\rangle}

\begin{document}

\htitle{2016}{Lab 5 Solutions}

\paragraph{Exercise 1:}
Consider the recursive \texttt{makeChange} implementation in the
lecture notes, without memoization:

\begin{verbatim}
# if we can't make change for n cents using L, returns -2
def makeChange(n, L):
    if n==0:
        return 0
    # if possible to make change for n, the answer is definitely less
    # than n + 1
    answer = n + 1
    for x in L:
        if x <= n:
            val = makeChange(n-x, L)
            if val != -2:
                answer = min(answer, 1 + val)
    if answer == n + 1:
        answer = -2
    return answer
\end{verbatim}

What's the running time of this function in terms of $n$ when L is
the list [1,2]?

\medskip

\textbf{Hint:} You've already figured out the answer
to this in a previous lab.

\medskip

\ans{Answer: $O(((1+\sqrt{5})/2)^n)$}.  The recurrence for the running time is
exactly the same in the case as it is for the recursive implementation
of the \texttt{fibonacci} function ($T(n) = T(n-1) + T(n-2)$ for $n\ge
2$).

\paragraph{Exercise 2:}
Modify the memoized \texttt{makeChange} implementation so that it
doesn't just tell you the minimum number of coins that are needed to
make change, but instead it returns a list of {\em which} coins you
should use to make change using the fewest number of coins.

\paragraph{Example solution:}
\begin{verbatim}
def memMakeChange(n, L, mem, mem2):
    if n==0:
        return 0
    elif mem[n] != -1:
        return mem[n]
    mem[n] = n
    for coin in L:
        if coin <= n:
            val = memMakeChange(n-coin, L, mem, mem2)
            if val+1 < mem[n]:
                mem[n] = val+1
                mem2[n] = coin
    return mem[n]

# mem[n] is the minimum number of coins to make change for n
# mem2[n] is one coin you can take as part of the optimal solution for n
def makeChange(n, L):
    mem = [-1]*(n+1)
    mem2 = [-1]*(n+1)
    memMakeChange(n, L, mem, mem2)
    answer = []
    while n > 0:
        answer += [mem2[n]]
        n -= mem2[n]
    return answer
\end{verbatim}

\paragraph{Exercise 3:}
Write a function \texttt{lis}(L) which takes as input a list of
integers L and outputs the length of the longest increasing
subsequence (lis) of L.  A subsequence of L is a sublist of L that
does not have to be contiguous.  For example, [1, 5, 9] is a
subsequence of the list L = [1,2,3,4,5,6,7,8,9] since 1,5,9 appear in
the list L in the same order (though just not in a row).  9,5,1 is not
a subsequence of L since it does not appear in L in that order.

Your implementation should run in time $O(m^2)$ where the size of L is
$m$.  \textbf{Bonus (try at the end):} See if you can find a way to
solve the problem in $O(m\log m)$ time.

\paragraph{Example solution:}
Here is a solution running in time $\Theta(m^2)$. The basic idea is
to let $f(\mathrm{last}, \mathrm{at})$ be the length of the longest
increasing
subsequence where the first number must be larger than L[last] and
you're only allowed to use the numbers L[at],L[at+1],$\ldots$.  Then,
$f$ can be computed recursively, and the answer is $f(i, i+1)$ for
some $i$ (since the sequence has to start with {\em some} value L[i]),
so the best choice over all $i$ should be returned.
Memoization is used to make it faster.

\begin{verbatim}
def memlis(L, last, at, mem):
    if at==len(L):
        return 0
    elif mem[last][at]!=-1:
        return mem[last][at]
    mem[last][at] = memlis(L, last, at+1, mem)
    if L[at]>L[last]:
        mem[last][at] = max(mem[last][at], 1 + memlis(L, at, at+1, mem))
    return mem[last][at]

def lis(L):
    mem = []
    for i in xrange(len(L)):
        mem += [[-1]*len(L)]
    answer = 0
    for i in xrange(len(L)):
        answer = max(answer, 1 + memlis(L, i, i+1, mem))
    return answer
\end{verbatim}

\paragraph{Exercise 4:}
(taken from http://people.csail.mit.edu/bdean/6.046/dp/)

\medskip

You are given a list of boxes, each having some height, width, and
depth.  You want to stack boxes to make the tallest stack possible,
where you can only put one box on top of the other if its base has
strictly smaller length and width than the box immediately underneath
it.  Given a list L of boxes of the form L = [[width1, height1,
depth1], [width2, height2, depth2], $\ldots$, [widthm, heightm,
depthm]], implement a function \texttt{boxes}(L) which returns the
height of the tallest stack you can possibly make using the given
boxes.  You do not have to use all the boxes given to you, and you can
use the same box multiple times.

\paragraph{Example solution:}
\begin{verbatim}
# L is given in decreasing order of base area, and memoize(L, at, last, mem)
# is the tallest stack we can make using only the boxes L[at],L[at+1],...
# where the last box we used was L[last]
def memoize(L, at, last, mem):
    if at==len(L):
        return 0
    elif mem[at][last] != -1:
        return mem[at][last]
    mem[at][last] = memoize(L, at+1, last, mem)
    if L[at][0]<L[last][0] and L[at][1]<L[last][1]:
        mem[at][last] = max(mem[at][last], L[at][2] + memoize(L, at+1, at, mem))
    return mem[at][last]

def boxes(L):
    T = []
    # put all 6 rotations of each box in a list
    for box in L:
        a1 = [box[0]*box[1], box[0], box[1], box[2]]
        a2 = [box[0]*box[1], box[1], box[0], box[2]]
        b1 = [box[0]*box[2], box[0], box[2], box[1]]
        b2 = [box[0]*box[2], box[2], box[0], box[1]]
        c1 = [box[1]*box[2], box[1], box[2], box[0]]
        c2 = [box[1]*box[2], box[2], box[1], box[0]]
        T += [a1, a2, b1, b2, c1, c2]
    # sort the list in decreasing order of base area
    T.sort()
    T.reverse()
    L = []
    for x in T:
        L += [[x[1], x[2], x[3]]]
    mem = []
    for i in xrange(len(L)):
        mem += [[-1]*len(L)]
    ans = 0
    # try all possibilities for the box at the very bottom and
    # take the best one
    for i in xrange(len(L)):
        ans = max(ans, L[i][2] + memoize(L, i+1, i, mem))
    return ans
\end{verbatim}

\paragraph{Exercise 5:}
In the {\em knapsack} problem we assume we have a bag that can hold
some $n$ liters of volume, and we want to pack the bag with the most
valuable combination of items possible out of some given list of
items.  Implement a function
\texttt{knapsack}(n, L) which takes as input this bag volume n, and a
list of items L, and returns the maximum value you can pack in the
bag.  L is a list of lists, where each list in L is of size 2
containing the volume of the item as its 0th element, and its volume
as its 1st element.  For example, L = [[7, 10], [5, 6], [4, 5]]
represents a list of 3 items.  The first item takes 7 liters and is
worth 10 dollars, the second item takes 5 liters and is worth 6
dollars, and the last item takes 4 liters and is worth 5 dollars.
\texttt{knapsack}(10, [[7, 10], [5, 6], [4, 5]]) should return $11$
since the best thing to do is to take the and and third item (which
both fit, since their total volume is $5+4=9$ liters, and we can fit
10 liters).

Your implementation should run in time $O(nm)$ where the size of L is
$m$.

\paragraph{Example solution:}
The basic idea is to let $f(\mathrm{n}, \mathrm{at})$ be the most
value you can get in a bag of size $n$ using only the items
L[at],L[at+1],$\ldots$.  $f$ can be computed recursively, and
memoization makes it faster.  Then the answer we finally want is $f(n,
0)$.

\begin{verbatim}
def memKnapsack(L, n, at, mem):
    if at==len(L):
        return 0
    elif mem[n][at]!=-1:
        return mem[n][at]
    mem[n][at] = memKnapsack(L, n, at+1, mem)
    if L[at][0]<=n:
        mem[n][at] = max(mem[n][at], L[at][1] + memKnapsack(L, n-L[at][0], at+1, mem))
    return mem[n][at]    

def knapsack(n, L):
    mem = []
    for i in xrange(n+1):
        mem += [[-1]*len(L)]
    return memKnapsack(L, n, 0, mem)
\end{verbatim}

\end{document}