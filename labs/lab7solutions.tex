%%
%% Style file "borrowed" from Michel Goemans
%%
\documentclass[11pt]{article}

\usepackage{url,amsmath,amsthm,amsfonts,amssymb}
\usepackage[usenames]{color}
\newcommand{\ans}[1]{\textcolor{red}{#1}}

\def\ceil#1{\lceil #1 \rceil}
\def\floor#1{\lfloor #1 \rfloor}
\def\ang#1{\langle #1 \rangle}

\newtheorem{definition}{Definition}
\newtheorem{remark}{Remark}
\newtheorem{theorem}{Theorem}
\newtheorem{lemma}[theorem]{Lemma}
\newtheorem{corollary}[theorem]{Corollary}
\newtheorem{proposition}[theorem]{Proposition}
\newtheorem{claim}[theorem]{Claim}
\newtheorem{observation}{Observation}

\newcommand{\R}{{\mathbb R}}
\newcommand{\Var}{\hbox{Var}}
\newcommand{\Z}{{\mathbb Z}}
\newcommand{\Q}{{\mathbb Q}}
\newcommand{\C}{{\mathbb C}}
\newcommand{\N}{{\mathbb N}}

\newlength{\toppush}
\setlength{\toppush}{2\headheight}
\addtolength{\toppush}{\headsep}

\newcommand{\htitle}[2]{\noindent\vspace*{-\toppush}\newline\parbox{6.5in}
 {\large Sports Academy \hfill #1\newline
\hspace*{\fill}{\bf Algorithms and Programming for High Schoolers} \hspace*{\fill} \newline
\mbox{}\hrulefill\mbox{}}\vspace*{1ex}\mbox{}\newline
\begin{center}{\Large\bf #2}\end{center}}

\newcommand{\handout}[3]{\thispagestyle{empty}
 \markboth{ #2}{ #2}
 \pagestyle{myheadings}\htitle{\protect\ref{#1}}{#2}{#3}}

\setlength{\oddsidemargin}{0pt}
\setlength{\evensidemargin}{0pt}
\setlength{\textwidth}{6.5in}
\setlength{\topmargin}{0in}
\setlength{\textheight}{8.5in}

\newcommand{\eps}{\varepsilon}
\newcommand{\sq}[1]{\langle#1\rangle}

\begin{document}

\htitle{2016}{Lab 7 Solutions}

\paragraph{Exercise 1:}
Write a function \texttt{fromListToMatrix}(A) which takes an adjacency
list representation of a graph A and outputs an adjacency matrix
representation of the same graph.  Similarly write a function
\texttt{fromMatrixToList}(A) which goes in the other direction.

\paragraph{Example solution:}
\begin{verbatim}
def fromListToMatrix(A):
    ans = []
    for i in xrange(len(A)):
        ans += [[False]*len(A)]
    for i in xrange(len(A)):
        for x in A[i]:
            ans[i][x] = True
    return ans

def fromMatrixToList(A):
    ans = []
    for i in xrange(len(A)):
        ans += [[]]
    for i in xrange(len(A)):
        for j in xrange(len(A)):
            if A[i][j]:
                ans[i] += [j]
    return ans
\end{verbatim}

\paragraph{Exercise 2:}
Write a function \texttt{connected}(A, i, j) which takes an adjacency
list A and two vertices i,j, and outputs \texttt{True} if there is a
path connecting i to j in the graph represented by A, and outputs
\texttt{False} otherwise.

\paragraph{Example solution:}
\paragraph{Using DFS}

\begin{verbatim}
def visit(x, stack, visited):
    visited[x] = True
    stack += [x]

def connected(A, i, j):
    visited = [False]*len(A)
    stack = []
    visit(i, stack, visited)
    while len(stack) > 0:
        x = stack.pop()
        if x == j:
            return True
        for y in A[x]:
            if not visited[y]:
                visit(y, stack, visited)
    return False
\end{verbatim}

\paragraph{Using DFS via Recursion}

\begin{verbatim}
def visit(A, i, j, visited):
    if i == j:
        return True
    visited[i] = True
    for k in A[i]:
        if (not visited[k]) and visit(A, k, j, visited):
            return True
    return False

def connected(A, i, j):
    visited = [False]*len(A)
    return visit(A, i, j, visited)
\end{verbatim}

\paragraph{Using BFS}
\begin{verbatim}
from collections import deque

def visit(x, queue, visited):
    visited[x] = True
    queue += [x]

def connected(A, i, j):
    visited = [False]*len(A)
    queue = deque()
    visit(i, queue, visited)
    while len(queue) > 0:
        x = queue.popleft()
        if x == j:
            return True
        for y in A[x]:
            if not visited[y]:
                visit(y, queue, visited)
    return False
\end{verbatim}

\paragraph{Exercise 3:}
Write a function \texttt{path}(A, i, j) which takes an adjacency
list A and two vertices i,j, and outputs a path from i to j
in the graph represented by A in \texttt{list} form, or [] if no such
path exists.  For example, if we can get from vertex 0 to 5 by taking
the sequence of edges $(0,1), (1, 4), (4, 7), (7,5)$ in some graph,
then the function should return the \texttt{list} [0, 1, 4, 7, 5].

\paragraph{Using DFS}

\begin{verbatim}
def visit(x, last, stack, visited, previous):
    visited[x] = True
    previous[x] = last
    stack += [x]

def path(A, i, j):
    visited = [False]*len(A)
    # previous[k] is the vertex right before k on a path from i to k
    previous = [-1]*len(A)
    stack = []
    visit(i, -1, stack, visited, previous)
    while len(stack) > 0:
        x = stack.pop()
        if x == j:
            ans = [j]
            while ans[len(ans)-1] != i:
                ans += [previous[ans[len(ans)-1]]]
            ans.reverse()
            return ans
        for y in A[x]:
            if not visited[y]:
                visit(y, x, stack, visited, previous)
    return []
\end{verbatim}

\paragraph{Using DFS via Recursion}

\begin{verbatim}
def visit(A, i, j, visited, previous):
    if i == j:
        return True
    visited[i] = True
    for k in A[i]:
        if not visited[k]:
            previous[k] = i
            if visit(A, k, j, visited, previous):
                return True
    return False

def path(A, i, j):
    visited = [False]*len(A)
    previous = [-1]*len(A)
    if not visit(A, i, j, visited, previous):
        return []
    ans = [j]
    while ans[len(ans)-1] != i:
        ans += [previous[ans[len(ans)-1]]]
    ans.reverse()
    return ans
\end{verbatim}

\paragraph{Using BFS}
\begin{verbatim}
from collections import deque

def visit(x, last, queue, visited, previous):
    visited[x] = True
    previous[x] = last
    queue += [x]

def path(A, i, j):
    visited = [False]*len(A)
    previous = [-1]*len(A)
    queue = deque()
    visit(i, -1, queue, visited, previous)
    while len(queue) > 0:
        x = queue.popleft()
        if x == j:
            ans = [j]
            while ans[len(ans)-1] != i:
                ans += [previous[ans[len(ans)-1]]]
            ans.reverse()
            return ans
        for y in A[x]:
            if not visited[y]:
                visit(y, x, queue, visited, previous)
    return []
\end{verbatim}

\paragraph{Exercise 4:}
Write a function \texttt{numComponents}(A) which takes an adjacency
list A and outputs the number of connected components in the graph
represented by A.

\paragraph{Example solution:}
\paragraph{Using DFS}

\begin{verbatim}
def visit(x, stack, visited):
    visited[x] = True
    stack += [x]

def numComponents(A):
    visited = [False]*len(A)
    ans = 0
    for i in xrange(len(A)):
        if not visited[i]:
            ans += 1
            stack = []
            visit(i, stack, visited)
            while len(stack) > 0:
                x = stack.pop()
                for y in A[x]:
                    if not visited[y]:
                        visit(y, stack, visited)
    return ans
\end{verbatim}

\paragraph{Using DFS via Recursion}

\begin{verbatim}
def visit(A, i, visited):
    visited[i] = True
    for j in A[i]:
        if not visited[j]:
            visit(A, j, visited)

def numComponents(A):
    visited = [False]*len(A)
    ans = 0
    for i in xrange(len(A)):
        if not visited[i]:
            ans += 1
            visit(A, i, visited)
    return ans
\end{verbatim}

\paragraph{Using BFS}
\begin{verbatim}
from collections import deque

def visit(x, queue, visited):
    visited[x] = True
    queue += [x]

def numComponents(A):
    visited = [False]*len(A)
    ans = 0
    for i in xrange(len(A)):
        if not visited[i]:
            ans += 1
            queue = deque()
            visit(i, queue, visited)
            while len(queue) > 0:
                x = queue.popleft()
                for y in A[x]:
                    if not visited[y]:
                        visit(y, queue, visited)
    return ans
\end{verbatim}

\paragraph{Exercise 5:}
Write a function \texttt{distance}(A, i, j) which takes an adjacency
list A and two vertices i,j, and outputs the length of the shortest
path from i to j in the graph represented by A.  This can be done by
modifying the code for bfs to keep track of distances.

\paragraph{Example solution:}

\begin{verbatim}
from collections import deque

# D is the distance to x
def visit(x, queue, distance, D):
    distance[x] = D
    queue += [x]

# returns -1 if there's no path from i to j
# otherwise returns the distance
def distance(A, i, j):
    distance = [-1]*len(A)
    queue = deque()
    visit(i, queue, distance, 0)
    while len(queue) > 0:
        x = queue.popleft()
        if x == j:
            return distance[j]
        for y in A[x]:
            if distance[y] == -1:
                visit(y, queue, distance, distance[x] + 1)
    return -1
\end{verbatim}

\end{document}