%%
%% Style file "borrowed" from Michel Goemans
%%
\documentclass[11pt]{article}

\usepackage{url,amsmath,amsthm,amsfonts,amssymb}

\def\ceil#1{\lceil #1 \rceil}
\def\floor#1{\lfloor #1 \rfloor}
\def\ang#1{\langle #1 \rangle}

\newtheorem{definition}{Definition}
\newtheorem{remark}{Remark}
\newtheorem{theorem}{Theorem}
\newtheorem{lemma}[theorem]{Lemma}
\newtheorem{corollary}[theorem]{Corollary}
\newtheorem{proposition}[theorem]{Proposition}
\newtheorem{claim}[theorem]{Claim}
\newtheorem{observation}{Observation}

\newcommand{\R}{{\mathbb R}}
\newcommand{\Var}{\hbox{Var}}
\newcommand{\Z}{{\mathbb Z}}
\newcommand{\Q}{{\mathbb Q}}
\newcommand{\C}{{\mathbb C}}
\newcommand{\N}{{\mathbb N}}

\newlength{\toppush}
\setlength{\toppush}{2\headheight}
\addtolength{\toppush}{\headsep}

\newcommand{\htitle}[2]{\noindent\vspace*{-\toppush}\newline\parbox{6.5in}
 {\large Addis Ababa University, Amist Kilo \hfill #1\newline
\hspace*{\fill}{\bf Algorithms and Programming for High Schoolers} \hspace*{\fill} \newline
\mbox{}\hrulefill\mbox{}}\vspace*{1ex}\mbox{}\newline
\begin{center}{\Large\bf #2}\end{center}}

\newcommand{\handout}[3]{\thispagestyle{empty}
 \markboth{ #2}{ #2}
 \pagestyle{myheadings}\htitle{\protect\ref{#1}}{#2}{#3}}

\setlength{\oddsidemargin}{0pt}
\setlength{\evensidemargin}{0pt}
\setlength{\textwidth}{6.5in}
\setlength{\topmargin}{0in}
\setlength{\textheight}{8.5in}

\newcommand{\eps}{\varepsilon}
\newcommand{\sq}[1]{\langle#1\rangle}

\begin{document}

\htitle{July 18, 2016}{Lab 1}

Today you will gain some hands-on experience in programming by
working with Python using an application called Jupiter Notebook.  At your machine, run Jupyter Notebook by going to
Start$\rightarrow$All Programs$\rightarrow$Jupyter Notebook. From this interface, you can evaluate expressions.

If you are working on your own laptop, you should download Python 2.7
at \texttt{http://www.python.org}.  Also, the lecture notes and labs
are on the website \texttt{http://www.addiscoder.com}

\paragraph{Exercise 1:}
Imagine evaluating the expressions below in the order given.
What would they evaluate to (some of them will give
errors)? Come up with an answer for
each expression before typing it into Jupyter and see if you get it
right.

\begin{itemize}
\item[$>>>$] \begin{verbatim}5 + 4*3\end{verbatim}
\item[$>>>$] \begin{verbatim}5+4 * 3\end{verbatim}
\item[$>>>$] \begin{verbatim}(5 + 4)*3\end{verbatim}
\item[$>>>$] \begin{verbatim}(5 + 4.0)*3\end{verbatim}
\item[$>>>$] \begin{verbatim}(5 + 4.0)/3\end{verbatim}
\item[$>>>$] \begin{verbatim}5 + 4.0/3\end{verbatim}
\item[$>>>$] \begin{verbatim}5 + 4.0//3\end{verbatim}
\item[$>>>$] \begin{verbatim}5+`4' * 3\end{verbatim}
\item[$>>>$] \begin{verbatim}`5'+`4' * 3\end{verbatim}
\item[$>>>$] \begin{verbatim}(`5'+`4') * 3\end{verbatim}
\item[$>>>$] \begin{verbatim}2**3*3\end{verbatim}
\item[$>>>$] \begin{verbatim}(2**3)*3\end{verbatim}
\item[$>>>$] \begin{verbatim}2**(3*3)\end{verbatim}
\item[$>>>$] \begin{verbatim}[5]+`4' * 3\end{verbatim}
\item[$>>>$] \begin{verbatim}[5]+[4] * 3\end{verbatim}
\item[$>>>$] \begin{verbatim}[`a',5,2.0][2]\end{verbatim}
\item[$>>>$] \begin{verbatim}x = [`a',5,2.0]\end{verbatim}
\item[$>>>$] \begin{verbatim}x[0]\end{verbatim}
\item[$>>>$] \begin{verbatim}x[1]\end{verbatim}
\item[$>>>$] \begin{verbatim}x[1:2]\end{verbatim}
\item[$>>>$] \begin{verbatim}x[0:3]\end{verbatim}
\item[$>>>$] \begin{verbatim}x[0]*x[1]\end{verbatim}
\item[$>>>$] \begin{verbatim}x[0]*x[2]\end{verbatim}
\item[$>>>$] \begin{verbatim}x[1]+x[2]\end{verbatim}
\item[$>>>$] \begin{verbatim}x[1]**x[2]\end{verbatim}
\item[$>>>$] \begin{verbatim}x += x[1]*x\end{verbatim}
\item[$>>>$] \begin{verbatim}x\end{verbatim}
\item[$>>>$] \begin{verbatim}y = x[2:4]\end{verbatim}
\item[$>>>$] \begin{verbatim}y\end{verbatim}
\item[$>>>$] \begin{verbatim}x = x[0:2]\end{verbatim}
\item[$>>>$] \begin{verbatim}x + y\end{verbatim}
\item[$>>>$] \begin{verbatim}x = `123456789'\end{verbatim}
\item[$>>>$] \begin{verbatim}x + y\end{verbatim}
\item[$>>>$] \begin{verbatim}x[2:5]\end{verbatim}
\item[$>>>$] \begin{verbatim}x[2:5]*x[2]\end{verbatim}
\item[$>>>$] \begin{verbatim}y = [3,2,1]\end{verbatim}
\item[$>>>$] \begin{verbatim}x[2:5] + x[y[2]] \end{verbatim}
\item[$>>>$] \begin{verbatim}y = [1,2,3,4,5,6]\end{verbatim}
\item[$>>>$] \begin{verbatim}y\end{verbatim}
\item[$>>>$] \begin{verbatim}y[0:4] = [7]\end{verbatim}
\item[$>>>$] \begin{verbatim}y\end{verbatim}
\item[$>>>$] \begin{verbatim}not True or True\end{verbatim}
\item[$>>>$] \begin{verbatim}not (True or True)\end{verbatim}
\item[$>>>$] \begin{verbatim}x = 10\end{verbatim}
\item[$>>>$] \begin{verbatim}x>20 or x%5=1\end{verbatim}
\item[$>>>$] \begin{verbatim}x>20 or x%5==1\end{verbatim}
\item[$>>>$] \begin{verbatim}x<20 or x%5==0\end{verbatim}
\end{itemize}

To do the remaining exercises, log into \texttt{https://cs124.seas.harvard.edu}.

\end{document}
