\documentclass[11pt]{article}

\usepackage{url,amsmath,amsthm,amsfonts,amssymb}
\usepackage[usenames]{color}
\newcommand{\ans}[1]{\textcolor{red}{#1}}

\def\ceil#1{\lceil #1 \rceil}
\def\floor#1{\lfloor #1 \rfloor}
\def\ang#1{\langle #1 \rangle}

\newtheorem{definition}{Definition}
\newtheorem{remark}{Remark}
\newtheorem{theorem}{Theorem}
\newtheorem{lemma}[theorem]{Lemma}
\newtheorem{corollary}[theorem]{Corollary}
\newtheorem{proposition}[theorem]{Proposition}
\newtheorem{claim}[theorem]{Claim}
\newtheorem{observation}{Observation}

\newcommand{\R}{{\mathbb R}}
\newcommand{\Var}{\hbox{Var}}
\newcommand{\Z}{{\mathbb Z}}
\newcommand{\Q}{{\mathbb Q}}
\newcommand{\C}{{\mathbb C}}
\newcommand{\N}{{\mathbb N}}

\newlength{\toppush}
\setlength{\toppush}{2\headheight}
\addtolength{\toppush}{\headsep}

\newcommand{\htitle}[2]{\noindent\vspace*{-\toppush}\newline\parbox{6.5in}
 {\large Sports Academy \hfill #1\newline
\hspace*{\fill}{\bf Algorithms and Programming for High Schoolers} \hspace*{\fill} \newline
\mbox{}\hrulefill\mbox{}}\vspace*{1ex}\mbox{}\newline
\begin{center}{\Large\bf #2}\end{center}}

\newcommand{\handout}[3]{\thispagestyle{empty}
 \markboth{ #2}{ #2}
 \pagestyle{myheadings}\htitle{\protect\ref{#1}}{#2}{#3}}

\setlength{\oddsidemargin}{0pt}
\setlength{\evensidemargin}{0pt}
\setlength{\textwidth}{6.5in}
\setlength{\topmargin}{0in}
\setlength{\textheight}{8.5in}

\newcommand{\eps}{\varepsilon}
\newcommand{\sq}[1]{\langle#1\rangle}

\begin{document}

\htitle{2016}{Lab 9 Solutions}

\paragraph{Exercise 1:}
The US coin denominations are $1$, $5$, $10$, and $25$ cents.  To
minimize the number of coins needed to make change for $n$ cents, it
is always optimal to repeatedly take the biggest coin possible.  Write
a function \texttt{makeChange}(n) which outputs the list of coins one
should use to make change for $n$ cents using US coins.

\paragraph{Example solution:}
\begin{verbatim}
def makeChange(n):
    if n == 0:
        return []
    elif n >= 25:
        return [25] + makeChange(n-25)
    elif n >= 10:
        return [10] + makeChange(n-10)
    elif n >= 5:
        return [5] + makeChange(n-5)
    else:
        return [1] + makeChange(n-1)
\end{verbatim}

\paragraph{Exercise 2:}
There are $n$ possible activities that you can do throughout the day.
Each activity has a start time $a_i$ and a finish time $b_i$.  You
cannot participate in two activities if they overlap.  Calculate the
most number of activities that you could participate in.  That is,
write a function \texttt{mostActivities}(L) where L is of the form
[[$a_1,b_1$], [$a_2,b_2$], $\ldots$, [$a_n, b_n$]], and where
\texttt{mostActivities}(L) outputs the largest number of activities
that you could participate in.

This problem can be solved using both memoization and the greedy
method.  What's the running time you get for each method?

\paragraph{Example solution:}
\begin{verbatim}
# greedy implementation
# sort by the b_i and greedily schedule to take the next activity
# which ends the earliest
def mostActivities(L):
    for x in L:
        x[0],x[1] = x[1],x[0]
    L.sort()
    for x in L:
        x[0],x[1] = x[1],x[0]

    ans = 1
    last = L[0][1]
    for x in L:
        if x[0] < last:
            continue
        else:
            ans += 1
            last = x[1]
    return ans
\end{verbatim}

\begin{verbatim}
# memoization implementation
# f(at, last) is the most number of activities that can be scheduled
# amongst activities L[at], L[at+1],... where the last activity
# scheduled was L[last] 
def f(L, at, last, mem):
    if at == len(L):
        return 0
    elif mem[at][last] != -1:
        return mem[at][last]
    mem[at][last] = f(L, at+1, last, mem)
    if L[at][0] >= L[last][1]:
        mem[at][last] = max(mem[at][last], 1 + f(L, at+1, at, mem))
    return mem[at][last]

def mostActivities(L):
    for x in L:
        x[0],x[1] = x[1],x[0]
    L.sort()
    for x in L:
        x[0],x[1] = x[1],x[0]

    mem = []
    for i in xrange(len(L)):
        mem += [[-1]*len(L)]
    return 1 + f(L, 1, 0, mem)
\end{verbatim}

The memoized recursive implementation takes $\Theta(n^2)$ time, where
$n$ is the length of $L$.  The greedy implementation takes
$\Theta(n\log n)$ time: the sort in the beginning dominates the
running time, since the rest of the algorithm takes only $O(n)$ time.

\paragraph{Exercise 3:}
You have a bag that can hold $W$ kilograms of stuff.  You're at a
grain store where you pay only by the total weight of items you buy
and not
by the total value of the grains, so you want to pack your bag as well
as possible
to maximize the total value of goods you take.  The store has a
separate container for each type of grain, and there are $w$ pounds
available of each grain, and those $w$ pounds in total are worth $v$
dollars.  The grains have all been powdered, so it is easy to take any
number of pounds of a grain which is at most $w$.  Write a function
\texttt{maxGrain}(L, W) which takes a list L = [[$v_1,w_1$], [$v_2,w_2$],
$\ldots$, [$v_n,w_n$]] of the values and weights of all the grain
containers, and outputs a number which is the maximum number of
dollars' worth of grain you can take out of the store in your
weight-$W$ bag.  Your output should be a \texttt{float}.  You can
convert an \texttt{int} x to a \texttt{float} in Python by either
multiplying it by 1.0, or typing \texttt{float}(x).

\paragraph{Example solution:}
The idea of the solution is to sort the items by $v_i/w_i$ in
decreasing order, then greedily take the items which are worth the
most value per kilogram.

\begin{verbatim}
def maxGrain(L, W):
    T = []
    for i in xrange(len(L)):
        L[i] = [1.*L[i][0]/L[i][1]] + L[i]
    L.sort()
    L.reverse()
    ans = 0
    for x in L:
        if W == 0:
            break
        ratio = x[0]
        v = x[1]
        w = x[2]
        take = min(w, W)
        ans += ratio*take
        W -= take
    return ans
\end{verbatim}

\end{document}